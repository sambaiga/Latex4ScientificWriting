% !TeX document-id = {97ff76cc-a53b-499b-868e-cac32905f45a}
%%%%%%%%%%%%%%%%%%%%%%%%%%%%%%%%%%%%%%%%%%%%%%%%%%%%%%%%%%%%
%%
%% LaTeX Thesis Template defined by Anthony Faustine(sambaiga@gmail.com), 2016
%%
%%%%%%%%%%%%%%%%%%%%%%%%%%%%%%%%%%%%%%%%%%%%%%%%%%%%%%
% !TeX TXS-program:compile = txs:///pdflatex/[--shell-escape]
\RequirePackage[l2tabu, orthodox]{nag}  %checks for obsolete LaTeX packages and outdated commands


\documentclass[a4paper,12pt]{article}
 
\usepackage{setspace} %set line spacing
\doublespacing
% or:
%\onehalfspacing
\setlength{\parindent}{0pt} %set noindent for all paragraphs
\setlength{\parskip}{2.0ex plus0.5ex minus0.2ex}


% package imports
\usepackage{amsmath,amsthm,amssymb,amsfonts} %for math environments

\usepackage{a4wide}  % Pulls the page out a bit - makes it look better (in my opinion)
\usepackage{parskip}  % Removes paragraph indentation (not needed most of the time now)
\usepackage{fullpage}
\usepackage{blindtext}
\usepackage{algorithm}
\usepackage[pdftex]{graphicx}
\usepackage{float}
\usepackage{subfig}
\usepackage{siunitx}  % for writing scientific documents, where units and numbers are a big part of the writing.
\usepackage{booktabs} % To create tables without vertical separators.
\usepackage[table,xcdraw]{xcolor}
\usepackage{multirow} %for table with multi column
\usepackage{color}
\usepackage{algorithm}
\usepackage{algorithmic} %for algorithm
\usepackage[round]{natbib} %for citation 
\usepackage[margin=10pt,font=small,labelfont=bf, labelsep=endash]{caption}% Vastly improves the standard formatting of captions
\usepackage{todonotes}
\usepackage[margin=1in]{geometry}
\usepackage{bm}
%\usepackage{apacite}
%\usepackage{subfigure} % Provides commands to make subfigures (figures with (a), (b) and (c))
\usepackage{pdflscape} %for land scapping pages
\usepackage[final]{microtype} %improves the spacing between words and letters
\usepackage{footnote}
\usepackage[colorlinks]{hyperref}
%\usepackage{pslatex}
\usepackage{times}
\usepackage{minted}
%\usemintedstyle{monokai}
\usepackage{cleveref}   

% settings
%\bibliographystyle{plain}
\setcounter{secnumdepth}{3}
\setcounter{tocdepth}{3}
\pagenumbering{roman}



\setlength{\textheight}{9.8in} \setlength{\topmargin}{0.0in}
\setlength{\headheight}{0.0in} \setlength{\headsep}{0.0in}
\setlength{\leftmargin}{0.0in}
\setlength{\oddsidemargin}{0.0in}
%\setlength{\parindent}{1pc}
\setlength{\textwidth}{6.5in}
%\linespread{1.6}

% Useful custom commands
\newcommand{\note}[1]{\color{blue}#1\color{black}}
\newcommand{\missref}{\note{[REF]}}


\newcommand{\etal}{\textit{et.~al.}}
\newcommand{\phys}{\textit{Physarum Polycephalum}}

% The document starts here
\title{{\Large \textbf{ {\color{green} \textbf{Workshop On LaTeX for Scientific Writing} } \\ Day 1: Activities}}}
\author{\href{sambaiga.github.io}{Anthony Faustine} \\sambaiga@gmail.com}
\date{14 August 2017}


  
\begin{document}
\maketitle
\newcounter{rom}
 

 
\addtocounter{rom}{1}\setcounter{page}{2}~


%\tableofcontents
%\listoffigures
%\listoftables
%\newpage\thispagestyle{plain}~

\pagenumbering{arabic}



\section*{{\color{green} Activity 1: LaTeX Hello word}}
\begin{enumerate}
	\item Create a new file in TexStudio say Activity.tex, that contains the following text and \LaTeX commands:
	\begin{minted}{tex}
	\documentclass{article}
	\begin{document}
	Hello world ! % This is your content
	This is a simple example to start with \LaTeX.
	\end{document}
	The first task.
	\end{minted}
	
	\item Run quick build.
\end{enumerate}


%%Activity 2
\section*{{\color{green} Activity 2: Document class}}
\begin{enumerate}
	\item Change the document class of Activity.tex from article into beamer.\\
	\mintinline{tex}{\documentclass{beamer}}
	\item  Run quick build. What do you see?
\end{enumerate}


%%Activity3
\section*{{\color{green} Activity 3: Document Tittle}}

\begin{enumerate}
	\item Change the document class of Activity.tex from beamer into article class with the a4paper and 12pt	options.\\
	\mintinline{tex}{\documentclass[a4paper,12pt]{article}}
	\item Create the title of your article and put two authors and date.
	\begin{minted}{tex}
	\title{Scintific Writing using LaTeX}
	\author{N.~Mduma \and S.~Mtey}
	\date{\today}
	\end{minted}
	\item Add \mintinline{tex}{\maketitle} just after \mintinline{tex}{\begin {document}}
	\item Run quick build. What do you see?.
	\item Use geometry packages to set document margin.\\
	\mintinline{tex}{\usepackage[top=1in,bottom=1in,left=1in,right=1in]{geometry}}
	\item Try to change the margin to different numbers, run quick build. What do you see?
	\item Try to change the margin to different unit such as centimetres (e.g \mintinline{tex}{top=2.5cm}), run quick build. What do you see?
\end{enumerate}


%%Activity4
\section*{{\color{green} Activity 4: Sections}}
\begin{enumerate}
	\item Open the Activity.tex and create a section structure like shown below:\\
		\textbf{1. Introduction}\\
			Hello world ! % This is your content
		This is a simple example to start with \LaTeX.\\
		 \textbf{2 Methods}\\
		 \textbf{2.1 Model} \\
		 \textbf{2.1.1 Model Assumption} \\
		 \textbf{3 Results} \\
		 \textbf{4 Conclusion}
  \item Run quick build. What do you see?
  \item What happens if you use the *-version e.g \mintinline{tex}{\subsection*{Results}}.
  \item Why is it not possible to use \mintinline{tex}{ \chapter {} }, in this document?.
\end{enumerate}


%%Activity Five
\section*{{\color{green} Activity 5: Text Formatting}}
\begin{enumerate}
	\item Bold the tittle of your Activity.tex and the font size to \mintinline{tex}{\Large}.
	\item Change the font style of author names to italic, use \mintinline{tex}{\textit{text}} 
	\item Change the text color of date to green.First add \mintinline{tex}{\usepackage{xcolor}} and then use \mintinline{tex}{\textcolor{green}{text}}
	\item Produce the following text in the Introduction section. \textbf{Hint}: in the itemize environment you can specify what character to use as bullet: \mintinline{tex}{\item[<optional character>]}
	\\
	{\LARGE H}ello, this is my first attempt at writing in \textrm{LaTeX}. {\color{blue}I'm hoping that once I've mastered LaTeX 100\%}, everyone will be so in awe
	of my beautiful papers \& books that they'll publish them straight away without
	all that \textit{boring nonsense with referees}. I haven't written very much yet but I think I'm starting to get the hang of it. And this is what I plan to do:
	\begin{itemize}
		\item Practise LaTeX in:
		\begin{enumerate}
			\item [*] Teaching
			\item [*] Writing books
			\item [*] etc
		\end{enumerate}
	\end{itemize}
	
\end{enumerate}

%% Activity 6

\section*{{\color{green} Activity 6: Cross-reference section}}
\begin{enumerate}
	\item Experiment with the section cross-reference in the Activity.tex
	\item Try writing \mintinline{tex}{\tableofcontents} in the top of your document just after \mintinline{tex}{\begin{document}} What happens when you typeset?
	\item Add \mintinline{tex}{\usepackage{cleveref}} and try to use \mintinline{tex}{\cref{key}} command instead of \mintinline{tex}{\ref{key}}	command. What happens? Do there any difference between \mintinline{tex}{\cref{key}} and \mintinline{tex}{\ref{key}} 
\end{enumerate}
 


\section*{{\color{green} Activity 7: Math typesetting}}
Open the Activity.ex file and type the following under Model assumption subsection:
\begin{enumerate}
	\item  In this work we demonstrate that $\alpha^2 + \beta^2 \gg \frac{\pi}{4}$ is only correct if the Euler condition $\nabla x =0$ is satisfied. \textbf{Hint}: To typeset $\gg$ use \mintinline{tex}{\gg} command.
	
	\item We propose a new numerical approach to solve the time-dependent Schr\"odinger equation as shown in \eqref{odinger};
	\begin{equation}\label{odinger} i\hbar \frac{\partial
		\Psi(t)}{\partial t} = H(t) \Psi(t)
	\end{equation}
	where $i$ is the imaginary unit, $\hbar$ is the reduced Planck constant, the symbol $\frac{\partial }{\partial t}$ indicates a partial derivative with respect to time $t$. \textbf{Hint}: To typeset symbols $\hbar$ and $\Psi$ use the following commands, \mintinline{tex}{\hbar} and \mintinline{tex}{\Psi} respectively. To typeset ancient symbol \"o use \mintinline{tex}{\"o} command.

\item The relation between the golden ratio and the Fibonacci series is given by \eqref{fib-series}.
\begin{equation}\label{fib-series}
\phi = 1 + \sum^{\infty} _{n=1}
\frac{ (-1)^{n+1} }{ F_n F_{n+1} }
\end{equation}
where the golden ratio $\phi =
\frac{1}{2} (1 + \sqrt{5})$
\item What happens if you use \mintinline{tex}{\ref{key}} instead of \mintinline{tex}{\eqref{key}} to reference an equation.
\item What happens if you use the *-version of equation environment \mint{tex}{\begin{equation*}...\end{equation*}}
\end{enumerate}

\section*{{\color{green} Activity 8: Math typesetting}}
Open the Activity.ex file and type the following under Model subsection:
\begin{enumerate}
	\item  Consider a narrowband point-to-point communication system of $M_t$ transmit and $M_r$ receive antennas. The received signal vector $y$ can be represented by the following discrete time model.
	\begin{equation*}
	\begin{bmatrix}
	y_1\\ \vdots \\y_{M_r} 
	\end{bmatrix}
	= \begin{bmatrix}
	h_{11} &\ldots & h_{1M_t}\\
	\vdots  & \ddots & \vdots\\
	h_{M_r1} &\ldots& h_{M_rM_t}
	\end{bmatrix}
	\begin{bmatrix}
	x_1\\ \vdots \\x_{M_t} 
	\end{bmatrix}
	+
	\begin{bmatrix}
	n_1\\ \vdots \\n_{M_r} 
	\end{bmatrix}
	\end{equation*}
	
\end{enumerate}

\end{document}