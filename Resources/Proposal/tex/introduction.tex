\section{Introduction}
Worldwide electrical energy consumption is constantly increasing especially in developing countries \citep{Monacchi2013}. According to United States Energy Information Administration report\footnote{http://www.eia.gov/todayinenergy/detail.cfm?id=14011}, \textit{the developing countries will account for 65\% of the global energy consumption by 2040}. 

\lipsum[1]

\subsection{Background}

Over the recently most part of the world has witnessed rapidly increasing to energy use in buildings (residential and commercial). Building contribute about 40\% of the total global energy \citep{Batra2014c}. Residential and commercial buildings consume approximately 60\% of the world’s electricity \footnote{The United Nation’s Environment Programme’s Sustainable Building and Climate Initiative (UNEP-SBCI)}. In U.S. A for example 74.9\% of all the electricity produced is used just to operate buildings \footnote{\href{http://www.eia.gov/todayinenergy/detail.cfm?id=14011}{United States Energy Information Administration report}} while in Africa, 56\% of the total electric power demand is from buildings \citep{Kitio2013}. Thus, energy saving in buildings will have significant impact on the reduction of overall energy consumption.

\lipsum[1]

\subsection{Problem Statement}

The key challenge to NILM problem is how to design efficient unsupervised NILM algorithm that can run in real-time using low-frequency sampling data. Several state-of-the-art NILM algorithms have been proposed using different approaches such as different variants of Hidden Markov Models (HMM) \citep{Kim2011,Parson2012,Kolter2012,Makonin2015}, Deep Neural Networks (DNN) \citep{Badayos2015,Paulo2016a}, Graph Signal Processing (GSP) \citep{Stankovic2014,Zhao2016a} and Combinatorial Optimization (CO) \citep{ReyesLua2015,Batra2013}. However, most of these algorithms suffer from high computational complexity which make them unstable for real-time applications, cannot be generalized across different buildings, requires lot of training data, their performance is limited to few numbers of appliances and are sensitive to noise and similar devices

\lipsum[1]
\subsection{Research Objectives}

\subsubsection{Main Objective}

The broad aim of this study is to develop NILM framework for sustainable residential buildings energy. The NILM framework will be well suited for the inherent characteristics of grids in Tanzania.

\subsubsection{Specific Objectives}
Specific Objectives are:
\begin{enumerate}
	\item To develop tools that will enable disaggregation research in developing countries.
	\item To develop innovative and real-time unsupervised NILM algorithms for sustainable energy consumption in residential buildings.
	\item To demonstrate and evaluate the potential of the proposed algorithm in (2) for sustainable
	energy consumption in residential buildings .
\end{enumerate}

\subsection{Research Questions}
This research is intended to answer the following questions:
\begin{enumerate}
	\item What tools can be developed to increase energy disaggregation research in developing countries?.
	\item How to design an efficient and real-time NILM algorithms that can be generalized across buildings  by taking into consideration developing countries characteristics?.
	\item Which and how innovative sustainable energy saving applications in residential buildings could be enabled with NILM algorithms in (2)?.
\end{enumerate}

\subsection{Significance of the Research}
The major contribution of the proposed study will be  novel unsupervised algorithm for energy disaggregation problem and its applicability in helping households to achieve quantifiable energy saving. The algorithm will provide real-time appliance specific information that will increase public awareness and make them be part and parcel of energy conservation. It will further help utility and policy makers gain better insights into energy consumption in residential buildings. 

The proposed study will establish tools and resource pertaining to energy consumption data sets. This will facilitate and promote research activities in energy disaggregation, energy data analysis, electricity grid modelling and appliance usage behaviour. Apart from that, energy consumption data sets  will be useful for policy makers in the energy sector.

The study will also contribute to the  understanding of challenges and possible strategies for energy conservation in residential buildings. Finally, the proposed study is expected to provide better theoretical understanding of NILM and its applicability in sustainable energy in residential buildings. 