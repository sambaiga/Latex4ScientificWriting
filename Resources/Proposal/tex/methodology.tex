\section{Research Methodology}
To achieve the research aim along with its objectives, this research will be grounded in design science research. Design science research is a problem-solving based research methodology that is focused on creation and investigation of technological artifacts  \citep{Hevner2004,Dresch2015,Wieringa2014}. It is oriented in solving a specific problems with the aim to obtain a solution that is liable to generalization for a specific class of problem even if the solution is not optimal \citep{Dresch2015}.

Since the main objective of this study is to develop NILM framework for sustainable energy in residential buildings, design science research gives the necessary framework for implementation of the proposed study. Through an iterative Vaishnavi and Kuechler's design cycle \citep{Tobergte2013}, the research will first create innovative artefacts which aims to improve energy efficiency in residential buildings and evaluate these concepts by providing concretes 
instantiation. 
\begin{figure}[ht]
	\centering
	\includegraphics[scale=0.48]{images/DSP1.pdf}
	\caption{Research Design}
	\label{fig:design_science}
\end{figure}

The research design is divided into five phases summarized in Figure \ref{fig:design_science}:

\textbf{Theory building:} The first phase of this study will be focused on literature review in the domain of NILM and its applicability in energy conservation in residential buildings. The purpose of literature review  to summarize the existing research and provide
future guidelines for research by identifying gaps in the existing literature \citep{Compton1993}. It allows the desired information to be extracted from an increasing volume of published results \citep{Seuring2012}. This will help to provide better understanding of the NILM problem, build a theoretical and practical framework related to a specific research questions and demarcate the scope of the study. For this study literature review will always be used as first step of knowledge acquisition.


\textbf{Empirical research:} In the second phase, building-block approach and experimental research will be used. The building-block approach will be employed in the development of a system that will facilitate collection of energy-data set in residential buildings. Building-block approach is a design method  for designing systems in which independently prepared modules are combined to form the final products \citep{Dutta2008}. The approach is suitable for this study as it allows for rapid prototyping and "try it and see" experimentation. 

The developed tools will be instrumented in one residential home in Tanzania  for one year. This will be used to gather ground truth electricity usage data from wide variety of loads as well the whole-house consumption at low sampling rate. The output of this experiment will lead to the establishment of the first energy datasets from Tanzania. The developed energy datasets will be technically validated using the Non-Intrusive Load Monitoring Toolkit (NILMTK)\footnote{http://nilmtk.github.io/} NILMTK. NILMTK is an open-source NILM toolkit written in Python and designed specifically to enable the comparison of NILM algorithms across diverse data sets \cite{Batra2014a,Kelly:2014:NVN}. Furthermore, the collected data will be analysed in order to gain insight into electrical energy usage behaviour, investigates pattern of energy use and will be used in the validation and evaluation of the proposed NILM algorithms and applications.

\textbf{Algorithms Development:} In this phase state-of-the arts NILM algorithms for energy disaggregation problem will be analysed and their performance evaluated in order to establish limitation and performance bound.~This will be followed by the design and development of NILM algorithms using creative methods and a well-established machine learning and signal processing theory found in the literature.~The algorithms will include a real-time NILM algorithm for residential energy disaggregation and innovative NILM  applications for sustainable energy consumption in residential buildings.~Then the developed NILM algorithms will be empirically validated over state of the art algorithms using real-energy dataset. NILMTK will be used to implement, test and evaluate the proposed NILM algorithms.

\textbf{Case Study:} In this phase a real application of the proposed algorithms to sustainable energy consumption in residential buildings will be considered. The objective of this case study will be to show how the proposed algorithms could help households in Tanzania achieve quantifiable energy saving and prove whether the proposed solutions works as expected when deployed in real grids setting. This will run on the real house data on embedded environment.

\textbf{Synthesis:} Finally the contribution, outcomes and findings of the entire research will be effectively communicated. It will include presentation of research findings in different scientific platforms(seminar,  conference and journals). The results will be presented both in technical audiences and to managerial audience. 