% !TeX document-id = {87d6827f-5870-46b7-927e-2f8eecb71b9d}
% !TeX TXS-program:compile = txs:///pdflatex/[--shell-escape]
\documentclass{beamer}
\usetheme[faculty=ped]{fibeamer}
\usepackage[utf8]{inputenc}
\usepackage{minted}
\usemintedstyle{monokai}
\usepackage{etoolbox}
\AtBeginEnvironment{minted}{\singlespacing%
	\fontsize{10}{10}\selectfont}
\usepackage[
  main=english, %% By using `czech` or `slovak` as the main locale
                %% instead of `english`, you can typeset the
                %% presentation in either Czech or Slovak,
                %% respectively.
  czech, slovak %% The additional keys allow foreign texts to be
]{babel}        %% typeset as follows:
%%
%%   \begin{otherlanguage}{czech}   ... \end{otherlanguage}
%%   \begin{otherlanguage}{slovak}  ... \end{otherlanguage}
%%
%% These macros specify information about the presentation
\title{ LaTex for Scientific Writing} %% that will be typeset on the
\subtitle{Day 2} %% title page.
\author{\href{sambaiga.github.io}{Anthony Faustine} \\sambaiga@gmail.com}


%% These additional packages are used within the document:
\usepackage{ragged2e}  % `\justifying` text
\usepackage{booktabs}  % Tables
\usepackage{tabularx}
\usepackage{tikz}      % Diagrams
\usepackage{siunitx}
\usepackage{etoolbox}
\usepackage[round]{natbib}
\preto\tabular{\shorthandoff{-}}
\usetikzlibrary{calc, shapes, backgrounds}
\usepackage{amsmath, amssymb}
\usepackage{url}       % `\url`s
\usepackage{listings}  % Code listings
\usepackage[colorinlistoftodos]{todonotes}
\let\todox\todo
\renewcommand\todo[1]{\todox[inline]{#1}}
%\setbeamercovered{highly dynamic}
\setbeamercovered{transparent}
\frenchspacing
\newcounter{saveenumi}
\newcommand{\seti}{\setcounter{saveenumi}{\value{enumi}}}
\newcommand{\conti}{\setcounter{enumi}{\value{saveenumi}}}

\resetcounteronoverlays{saveenumi}

\begin{document}
  \frame{\maketitle}

  \AtBeginSection[]{% Print an outline at the beginning of sections
    \begin{frame}<beamer>
      \frametitle{Outline}
      \tableofcontents[currentsection]
    \end{frame}}

  \begin{darkframes}
    


\section{Graphics, Figures and Tables}

\begin{frame}[fragile]{Tables}{Creating Tables}
Use the tabular environment
\begin{minted}{tex}
\begin{tabular}[position]{column alignments}
...
\end{tabular}
\end{minted}

\begin{description}
	\item[postion] is optional (vertical position): [t] (top), [c]
	(center, this is default), [b] (bottom);
	\item [column alignments]: l (left-justified), c (center justified),
	and r (right-justified);
\end{description}
\end{frame}

\begin{frame}[fragile]{Tables}{Creating Tables}
The column data is separated by \textcolor{blue}{\&},row end is marked as \mintinline{tex}{\\} and \mintinline{tex}{\hline} draw a horizontal line.\\
Consider the following simple table:\\


\begin{tabular}[c]{|l|l|}
	\hline 
	\textbf{Parameter} & \textbf{Value}\\
	\hline 
	Path loss (n) & 2.5\\
	Model & Okumura-model\\
	Cell-radius & $1km$\\
	\hline
\end{tabular}

\end{frame}

\begin{frame}[fragile]{Tables}{Creating Tables}
Latex code for previous simple table\\
	\begin{minted}{tex}

\begin{tabular}[t]{|l|l|}
\hline 
\textbf{Parameter} & \textbf{Value}\\
\hline 
Path loss (n) & 2.5\\
Model & Okumura-model\\
Cell-radius & $1km$\\
\hline
\end{tabular}
	\end{minted}
	
	
\end{frame}


\begin{frame}[fragile]{Tables}{Creating Tables}
The \alert{\texttt{booktabs }} package  improve the quality  LaTeX tables.
\begin{itemize}
	\item The horizontal rules are called with \mintinline{tex}{\toprule}, \mintinline{tex}{\midrule} and \mintinline{tex}{\bottomrule} instead of \mintinline{tex}{\hline} command.
	\item The \mintinline{tex}{\cmidrule} is used for mid-rules that span specified columns.
	\item The content of the tables is filled in the same manner as before.
	\item To use this package first you need to add this code in preamble. \mintinline{tex}{\usepackage{booktabs}}
\end{itemize}
\end{frame}

\begin{frame}[fragile]{Tables}{Creating Tables}
	\alert{Example:} 
	
	\centering
	\begin{tabular}[t]{ll}
		\toprule
		\textbf{Parameter} & \textbf{Value}\\
		\midrule
		Path loss (n) & 2.5\\
		Model & Okumura-model\\
		Cell-radius & $1km$\\
		\bottomrule
	\end{tabular}
	
\end{frame}

\begin{frame}[fragile]{Tables}{Creating Tables}
	\alert{Example:} consider the following code:\
	\begin{minted}{tex}
\centering
\begin{tabular}[t]{ll}
\toprule
\textbf{Parameter} & \textbf{Value}\\
\midrule
Path loss (n) & 2.5\\
Model & Okumura-model\\
Cell-radius & $1km$\\
\bottomrule
\end{tabular}
	\end{minted}
\end{frame}



\begin{frame}[fragile]{Tables}{Creating Tables}
To draw multicolumn table like this one:\\
\centering

\begin{tabular}{llr}
	\toprule
	\multicolumn{2}{c}{Name} \\
	\cmidrule(lr){1-2}
	First name	& Last Name	& Grade \\
	\midrule
	John 		    & Doe	  		& $7.5$ \\
	Richard 	  & Miles		  & $2$ \\
	\bottomrule
\end{tabular}
\end{frame}

\begin{frame}[fragile]{Tables}{Creating Tables}
	Use the following command: \mintinline{tex}{\multicolumn{n}{alignment}{item}}
	\begin{description}
		\item [n]: is the number of columns to be spanned.
		\item [alignemnt]: is one of the \mintinline{tex}{l, r and c}.
		\item [item]: is the content.
	\end{description}
\alert{Example:}
\begin{minted}[numbersep=5pt]{tex}

\multicolumn{2}{c}{Name} 

\end{minted}

\end{frame}

\begin{frame}[fragile]{Tables}{Creating Tables}

\begin{minted}{tex}
\begin{tabular}{llr}
\toprule
\multicolumn{2}{c}{Name} \\
\cmidrule(r){1-2}
First name & Last Name	& Grade \\
\midrule
John  & Doe	& $7.5$ \\
Richard   & Miles  & $2$ \\
\bottomrule
\end{tabular}
\end{minted}
	
\end{frame}

\begin{frame}[fragile]{Tables}{Floating Tables}
 Latex	provides the \texttt{table}  environments for typesetting floating tables.
 \begin{itemize}
 	\item A table environment is set up as follows:
 	\begin{minted}{tex}
\begin{table}
\caption{title}
\label{tab:xxx}
%Place the table here
\end{table}
 	\end{minted}
 \end{itemize}
\end{frame}

\begin{frame}[fragile]{Tables}{Floating Tables}
	\begin{description}
		\item [\mintinline{tex}{\caption}] command is optional and used to set table tittle.
		\item [\mintinline{tex}{\label}] command is also optional and is used to reference
		the table’s number.
	\end{description}
\alert{Example:} To produce the following table
\begin{table}
\caption{Simulation Parameters}
\label{tab:model_parameter}
	\begin{tabular}[t]{ll}
	\toprule
	\textbf{Parameter} & \textbf{Value}\\
	\midrule
	Path loss (n) & 2.5\\
	Model & Okumura-model\\
	Cell-radius & $1km$\\
	\bottomrule
\end{tabular}
\end{table}

\end{frame}


\begin{frame}[fragile]{Tables}{Floating Tables}
\begin{minted}{tex}
\begin{table}
\caption{Simulation Parameters}
\label{tab:model_parameter}
\begin{tabular}[t]{ll}
\toprule
\textbf{Parameter} & \textbf{Value}\\
\midrule
Path loss (n) & 2.5\\
Model & Okumura-model\\
Cell-radius & $1km$\\
\bottomrule
\end{tabular}
\end{table}
\end{minted}

\alert{Guide to Making nice Table}: \href{https://www.inf.ethz.ch/personal/markusp/teaching/guides/guide-tables.pdf}{click here}
\end{frame}

\begin{frame}[fragile]{Activity}
	\begin{center}
		{\Large \textbf{Activity 7}}
	\end{center}
	
\end{frame}


\begin{frame}[fragile]{Graphics}{Include Graphics}
The easiest way to include images in your document is to use
the \texttt{graphicx} package.\\
Load the package graphicx:\mintinline{tex}{ \usepackage{graphicx}}
\begin{itemize}
	\item The image format available depend on what you’re using to compile.
	\item If you’re compiling using pdflatex (recommended), then you
	can use jpg, png, pdf, or eps files.
	\item Place the file in the same directory as your tex file, and use the
	\mintinline{tex}{\includegraphics[key-values]{imagefile}} command.
\end{itemize}
\end{frame}

\begin{frame}[fragile]{Graphics}{Include Graphics}
	\alert{Example:} 
	\mintinline{tex}{\includegraphics[scale=0.2]{images/bulb}}
	\begin{center}
     \includegraphics[scale=0.2]{images/bulb}
	\end{center}
	
\end{frame}

\begin{frame}[fragile]{Graphics}{Include Graphics}
	The image can be scaled to a specified height and/or width as follows:
	\mintinline{tex}{\includegraphics[height=2in,width=1in]{images/bulb}}
	\begin{center}
	\includegraphics[height=1.5in,width=1in]{images/bulb}
	\end{center}
\end{frame}

\begin{frame}[fragile]{Graphics}{Floating images}
	Use \texttt{figure} environment:
	\begin{center}
		\begin{minted}{tex}
		\begin{figure}
			\includegraphics{file}
			\caption{title }
			\label{fig:xxx}
		\end{figure}
	\end{minted}
	\end{center}
\end{frame}



\begin{frame}[fragile]{Graphics}{Floating images}
		\begin{figure}
		\includegraphics[scale=0.15]{images/bulb}
		\caption{Green Bulb} 
		\label{fig:bulb}
		\end{figure}
	To print the list of figures and tables use \mintinline{tex}{ \listoffigures} and \mintinline{tex}{\listoftables}  respectively.\\		
	
\end{frame}

\section{Cross-references text}
\begin{frame}[fragile]{Cross-reference text}
	Use \mintinline{tex}{\label} and \mintinline{tex}{\ref} for automatic numbering.\\
	
	The amsmath package provides \mintinline{tex}{\eqref} for referencing equations. \\
	
	\alert{Example}
	
	\begin{minted}{tex}
\section{Introduction}\label{sec:intro}
In Section \ref{sec:method}, we \ldots
\section{Method}\label{sec:method}
	\end{minted}
	
\end{frame}

\begin{frame}[fragile]{Cross-reference text}
		\alert{Example:} Cross-reference figure and table.\\
		
Figure \ref{fig:bulb} show a green bulb. \begin{center}
	$\Downarrow$
\end{center} \mintinline{tex}{Figure \ref{fig:bulb} show a green bulb} \\

The simulation data are shown in table \ref{tab:model_parameter} \begin{center}
$\Downarrow$
\end{center} \mintinline{tex}{The simulation data are shown in table \ref{tab:parameter}} \\


	
\end{frame}

\begin{frame}[fragile]{Cross-reference text}
	\alert{Example: Cross-reference equation}
	
	\begin{equation} \label{eq:euler}
	e^{i\pi} + 1 = 0
	\end{equation}
	By \eqref{eq:euler}, we have \ldots
	\begin{center}
		$\Downarrow$
	\end{center}
\begin{minted}{tex}
\begin{equation} \label{eq:euler}
e^{i\pi} + 1 = 0
\end{equation}
By \eqref{eq:euler}, we have \ldots
\begin{center}
\end{minted}
\end{frame}

\begin{frame}[fragile]{Notes with todonotes}
The \mintinline{tex}{\todo} command from the \texttt{todonotes} package is great for leaving notes to yourself and your collaborators.\\

Include this package in preamble: \mintinline{tex}{\usepackage[colorinlistoftodos]{todonotes}}

\alert{Example:}\\
\mintinline{tex}{\todo{Plain todonotes.}} \\
	
\mintinline{tex}{\todo[color=blue!40]{Todonote with a different color.}}  

\end{frame}

\begin{frame}[fragile]{Notes with todonotes}
\begin{itemize}
	\item Only inline notes are supported with
	beamer, but margin notes are supported for normal
	documents.
	\item There is also a handy \mintinline{tex}{\listoftodos} command.
	\begin{itemize}
		\item To use this load: \mintinline{tex}{\usepackage[colorlinks]{hyperref}} before \mintinline{tex}{\usepackage[colorinlistoftodos]{todonotes}}
	\end{itemize}
\end{itemize}
	
\end{frame}

\section{Bibliography}

\begin{frame}[<+->]{Bibliography}
To manage and include references in a LATEX document use \textsc{BibTeX}.\\
\begin{description}[long text]
	\item[BibTex]: a bibliographic tool that is used with LaTeX to help organize the user's references and create a bibliography.
	\begin{itemize}
		\item A BibTeX user creates a bibliography file with \mintinline{tex}{.bib} extension.
		\item The \mintinline{tex}{.bib} file is called a BibTEX database $\Rightarrow$ a text file containing data in a structured format.
		\item Each entry in the \mintinline{tex}{.bib} file is formatted with a certain structure and is given a "key" by which the author can refer to it in the source file.
	\end{itemize}
\end{description}
\end{frame}

\begin{frame}[fragile]{Bibliography}{The BibTEX Format}
	The generic form of a BibTEX entry is
	\begin{center}
	\begin{minted}{tex}
@type{key, field1 = ‘‘ ’’ or {} or none,
 field2 = ‘‘ ’’ or {} or none, 
 ... 
 fieldn = ‘‘ ’’ or {} or none 
 }
	\end{minted}
\end{center}
\end{frame}

\begin{frame}[fragile]{Bibliography}{The BibTEX Format}
	\alert{Example:}
	\flushleft{
\begin{minted}{tex}
@article{Gettys90,
author = {Jim Gettys and Phil Karlton and Scott McGregor},
title = {The {X} Window System, Version 11},
journal = {Software Practice and Experience},
volume = {20},
number = {S2},
year = {1990},
abstract = {A technical overview of the X11 functionality.  This is an update
of the X10 TOG paper by Scheifler \& Gettys.}
}
\end{minted}
}
\end{frame}

\begin{frame}[<+->]{Bibliography}{Export .bib file from Mendeley}
\begin{itemize}
	\item Open Mendeley, and within "My Library" found on the left, select references that you would like to texport to BibTeX.
	\item In the drop-down menu in the toolbar at the top of the screen, click "File $\rightarrow$ Export.
	\item In the dropdown list of filetypes chose "BibTeX (*.bib)" and save to the same location as the LaTeX file.
\end{itemize}
\end{frame}


\begin{frame}[<+->]{Bibliography}{Auto-syncing from Mendeley to BibTeX}
	Mendeley has the built-in capability to auto sync a BibTeX file when changes have been made to your Mendeley library.\\
	To set up the Mendeley auto sync:
	\begin{itemize}
		\item Go to Mendeley Desktop preferences.
		\item Select the BibTeX tab. 
		\item Select the box labeled “Enable BibTeX syncing” and select the BiBTeX file option you prefer.
		\item Select the location where you want the generated .bib file(s) to be stored (this should be the same location as your LaTeX file(s)).
	\end{itemize}
\end{frame}

\begin{frame}[<+->]{Bibliography}{Using BibTEX in your LATEX Document}
	To use .bib file in  latex document:
	\begin{itemize}
		\item We can use \mintinline{tex}{natbib} packages with \mintinline{tex}{\citet and \citep} commands $\Rightarrow$ \href{http://merkel.texture.rocks/Latex/natbib.php}{Reference sheet for natbib usage}.
		\item Load with \mint{tex}{\usepackage[options]{natbib}} See list of  at the end of \href{http://merkel.texture.rocks/Latex/natbib.php}{Reference sheet for natbib usage}.
		\item Example: \mintinline{tex}{\usepackage[round]{natbib}}
		\item Include .bib file at the end of document with \mintinline{tex}{\bibliography{bib file}} and specify a bibliographic styles \mintinline{tex}{\bibliographystyle{stylename}}.
		
	\end{itemize}
\end{frame}


\begin{frame}[fragile]{Bibliography}{The BibTEX Format}
	\alert{Example:}
	
	\begin{minted}{tex}

	
	\bibliographystyle{apa} %\bibliographystyle{apacite} 
	\newpage
	\bibliography{bib/References_NILM}
		\end{minted}
\end{frame}

\begin{frame}[fragile]{Bibliography}{The BibTEX Format}
	\alert{Example:} 
	\begin{itemize}
		\item According to \citet{Barker2015} \ldots $\Rightarrow$ \mintinline{tex}{According to \cite{Alcala2015} \ldots}
		\item \ldots energy is important \citep{Barker2015}  $\Rightarrow$ \mintinline{tex}{\ldots energy is important \citep{Barker2015}}
	\end{itemize}	
\end{frame}

\section{Journal and Thesis}

\begin{frame}[<+->]{Thesis with Latex}
	\alert{Folder structure}: create a new folder (your project directory). 
	\begin{itemize}
		\item  Add some additional folders within this folder:
		\begin{description}
			\item [images] $\Rightarrow$ will contain all images.
			\item [tex] $\Rightarrow$ will contain \textsc{.tex} file.
			\item [bib] $\Rightarrow$ will contain bibliography files.
		\end{description}
	   \item This will help you keep the overview about your files.
	\end{itemize}
\end{frame}


\begin{frame}[fragile]{Thesis with Latex}
	\alert{Creating the main LaTeX document}: create a main tex file using document class report and save it into tex folder as Thesis.tex.
	\begin{minted}{tex}
\documentclass[a4paper,12pt]{report} 

\begin{document}

Hello World!

\end{document}
	\end{minted}
\end{frame}

\begin{frame}[fragile]{Thesis with Latex}
	\alert{Document structure}: Let’s create the main document structure for our thesis as shown below.
	\begin{columns}
		\begin{column}{0.5\textwidth}
	\begin{itemize}
		\item Title page
		\item Abstract
		\item Table of contents
		\item List of Algorithms
		\item List of Figures
		\item List of Tables
		\item Introduction
	\end{itemize}
     \end{column}
		\begin{column}{0.5\textwidth}
			\begin{itemize}
			\item Literature Review
		\item Research Methodology
		\item Discussion		
		\item Conclusion
		\item Acknowledgment
		\item Appendices
		\item Bibliography
	\end{itemize}
\end{column}
\end{columns}
\end{frame}

\begin{frame}[fragile]{Thesis with Latex}
	\begin{itemize}
		\item First create the table of contents inside the begin and end document: \mintinline{tex}{\tableofcontents} $\Rightarrow$ a headline with Content will appear.
		\item To organize your files, create a new .tex file for each chapter of your thesis.
		\begin{itemize}
			\item Lets create a new file and save it as Introduction.tex into the tex subfolder.
			\begin{minted}{tex}
\chapter{Introduction}
\lipsum[2]
\section{Motivation}
\lipsum[4]
\section{Problem Statement}
\lipsum[3]
\section{Objectives}
\lipsum[2]
			\end{minted}
		\end{itemize}
	\end{itemize}
\end{frame}

\begin{frame}[<+->]{Thesis with Latex}
	\begin{itemize}
		\item Import the file into you main document (Thesis.tex) after \mintinline{tex}{\tableofcontents} as: \mintinline{tex}{\chapter{Introduction}
\lipsum[2]
\section{Motivation}
\lipsum[4]
\section{Problem Statement}
\lipsum[3]
\section{Objectives}
\lipsum[2]} 
		\item Great, we just started creating our document structure. 
		\item Lets put all the other chapters in there as well.
		\mintinline{tex}{\chapter{Literature Review}
\lipsum[2]
}\\
		\mintinline{tex}{\chapter{Research Methodology}

\lipsum[2-4]
}\\
		\mintinline{tex}{\input{tex/Results}}\\
		\mintinline{tex}{\chapter{Conclusion and Future Works}
\lipsum[2-4]
}
		\item Don’t forget to create the files, \mintinline{tex}{\chapter{Some Name}}  and save them.
	\end{itemize}
\end{frame}

\begin{frame}[<+->]{Thesis with Latex}
	\begin{itemize}
		\item Next add the additional lists just after \mintinline{tex}{\tableofcontents}\\
		\mintinline{tex}{\listoffigures}\\
		\mintinline{tex}{\listoftables}\\
		\mintinline{tex}{\listofalgorithms}
		\item The list of algorithms needs package: \mintinline{tex}{\usepackage{algorithm2e}}
	\end{itemize}
More details on how to typeset algorithms \href{https://en.wikibooks.org/wiki/LaTeX/Algorithms}{here.} 
\end{frame}

\begin{frame}[<+->]{Thesis with Latex}
	\alert{Add Bibliography}
	\begin{itemize}
		\item Auto-sync your Mendely bibliography library to bib subfolder.
		\item Add it to your main tex file and define a bibliography style just before \mintinline{tex}{\end{document}} \\
		\mintinline{tex}{\bibliographystyle{apa}}\\
		\mintinline{tex}{\bibliography{bib/References}}
		\item Include the natbib package in preamble for citation \mintinline{tex}{\usepackage[round]{natbib}} 
		\item For example to cite, write in any of you chapter files: \mintinline{tex}{	We refer to \citet{Mvuma2016} for things you ...}
	\end{itemize}
 
\end{frame}

\begin{frame}[<+->]{Thesis with Latex}
	\alert{The title page}
	\begin{itemize}
		\item  Add a title page. Most universities require to use a predefined title page.
		\item Create a title.tex file and save it in tex subfolder. Define the title page as title page by inserting; \mintinline{tex}{\begin{titlepage}
				tittle page contents
		\end{titlepage}}
	\item Input the file just after \mintinline{tex}{\begin{document}} as follows:\\
		
\end{itemize}

\end{frame}

\begin{frame}[fragile]{Thesis with Latex}
	\alert{Sample Tittle Page}
	\begin{minted}[baselinestretch=1,fontsize=\footnotesize]{tex}
\begin{titlepage}
\centering
{\Large \textbf{ MAXIMIZING LTE PERFORMACE WITH MIMO SYSTEMS}} \\
\vspace{0.5in}
\includegraphics[width=0.4\textwidth]{images/logo} \\
\vspace{1in}
{\large \textbf {\textbf{BY}}} \\
\vspace{.5in}
{\large \textbf {\textbf{JAMES KALUMUNA}}} \\
\vspace{1in}
{\large A dissertation in (partial) fulfillment of the requirements for the degree of
Master of Science in Telecommunications Engineering of the University of Dodoma} \\
\vspace{0.5in}
{\large University of Dodoma } \\
\vspace{0.1in}
{\large Jun 2015}
\end{titlepage}

	\end{minted}
	
\end{frame}


\begin{frame}[fragile]{Thesis with Latex}
	\alert{Add certification Page}: Create a Certification.tex file and save it in tex subfolder, define it  by inserting:
	
	\begin{minted}[baselinestretch=1,fontsize=\footnotesize]{tex}
\begin{center}
{\large \textbf{CERTIFICATION}} \\
\vspace{2in}
\end{center}
The undersigned certify that they have read and hereby recommend 
for acceptance by the University of Dodoma dissertation entitled 
\textbf{Maximizing LTE performance with MIMO Systems} in fulfillment
of the requirements for the degree of \textbf{Master of Science in 
Telecommunications Engineering} of the University of Dodoma.	
	\end{minted}
	
Input the file just after \mintinline{tex}{\input{tex/tittle}}.	
\end{frame}


\begin{frame}[fragile]{Thesis with Latex}
	\alert{Add certification Page}\\
	Includes supervisors approval as follows
	
	\begin{minted}[baselinestretch=1,fontsize=\footnotesize]{tex}
	\vspace{1in}
	\centering
	Approved by: 
	\bigbreak
	
\noindent\begin{tabular}{ll}
\makebox[2.5in]{\hrulefill} & \makebox[2.5in]{\hrulefill}\\
Prof A.N Mvuma (First Supervisor) & Date\\[8ex]% adds space between the two sets of signatures
\makebox[2.5in]{\hrulefill} & \makebox[2.5in]{\hrulefill}\\
Dr. Hector Mongi (Second Supervisor) & Date\\
\end{tabular}
		
	\end{minted}	
\end{frame}

\begin{frame}[fragile]{Thesis with Latex}
Add empty pages between the title, abstract and table of contents and change the numbering to be roman.	The real Arabic page numbering of the thesis should start  with the first page of the introduction.
	\begin{itemize}
	\item First include the following commands into the preamble:
\begin{minted}[baselinestretch=1,fontsize=\footnotesize]{tex}
\setcounter{secnumdepth}{3}
 % determines up to what level the sectioning titles are numbered
\setcounter{tocdepth}{3}
\pagenumbering{roman}
\end{minted}
\item Then add \mintinline{tex}{\newcounter{rom}} just after the \mintinline{tex}{\begin{document}}.
	\end{itemize}
\end{frame}

\begin{frame}[fragile]{Thesis with Latex}
	\begin{itemize}
		\item Also add the following commands just before \mintinline{tex}{\tableofcontents}.
		\begin{minted}[baselinestretch=1,fontsize=\footnotesize]{tex}
		\addtocounter{rom}{1}\setcounter{page}{2}~
		\newpage\thispagestyle{plain}\setcounter{page}{3}
		\end{minted}
		\item Finally add the following command just before \mintinline{tex}{\section{Introduction}
Worldwide electrical energy consumption is constantly increasing especially in developing countries \citep{Monacchi2013}. According to United States Energy Information Administration report\footnote{http://www.eia.gov/todayinenergy/detail.cfm?id=14011}, \textit{the developing countries will account for 65\% of the global energy consumption by 2040}. 

\lipsum[1]

\subsection{Background}

Over the recently most part of the world has witnessed rapidly increasing to energy use in buildings (residential and commercial). Building contribute about 40\% of the total global energy \citep{Batra2014c}. Residential and commercial buildings consume approximately 60\% of the world’s electricity \footnote{The United Nation’s Environment Programme’s Sustainable Building and Climate Initiative (UNEP-SBCI)}. In U.S. A for example 74.9\% of all the electricity produced is used just to operate buildings \footnote{\href{http://www.eia.gov/todayinenergy/detail.cfm?id=14011}{United States Energy Information Administration report}} while in Africa, 56\% of the total electric power demand is from buildings \citep{Kitio2013}. Thus, energy saving in buildings will have significant impact on the reduction of overall energy consumption.

\lipsum[1]

\subsection{Problem Statement}

The key challenge to NILM problem is how to design efficient unsupervised NILM algorithm that can run in real-time using low-frequency sampling data. Several state-of-the-art NILM algorithms have been proposed using different approaches such as different variants of Hidden Markov Models (HMM) \citep{Kim2011,Parson2012,Kolter2012,Makonin2015}, Deep Neural Networks (DNN) \citep{Badayos2015,Paulo2016a}, Graph Signal Processing (GSP) \citep{Stankovic2014,Zhao2016a} and Combinatorial Optimization (CO) \citep{ReyesLua2015,Batra2013}. However, most of these algorithms suffer from high computational complexity which make them unstable for real-time applications, cannot be generalized across different buildings, requires lot of training data, their performance is limited to few numbers of appliances and are sensitive to noise and similar devices

\lipsum[1]
\subsection{Research Objectives}

\subsubsection{Main Objective}

The broad aim of this study is to develop NILM framework for sustainable residential buildings energy. The NILM framework will be well suited for the inherent characteristics of grids in Tanzania.

\subsubsection{Specific Objectives}
Specific Objectives are:
\begin{enumerate}
	\item To develop tools that will enable disaggregation research in developing countries.
	\item To develop innovative and real-time unsupervised NILM algorithms for sustainable energy consumption in residential buildings.
	\item To demonstrate and evaluate the potential of the proposed algorithm in (2) for sustainable
	energy consumption in residential buildings .
\end{enumerate}

\subsection{Research Questions}
This research is intended to answer the following questions:
\begin{enumerate}
	\item What tools can be developed to increase energy disaggregation research in developing countries?.
	\item How to design an efficient and real-time NILM algorithms that can be generalized across buildings  by taking into consideration developing countries characteristics?.
	\item Which and how innovative sustainable energy saving applications in residential buildings could be enabled with NILM algorithms in (2)?.
\end{enumerate}

\subsection{Significance of the Research}
The major contribution of the proposed study will be  novel unsupervised algorithm for energy disaggregation problem and its applicability in helping households to achieve quantifiable energy saving. The algorithm will provide real-time appliance specific information that will increase public awareness and make them be part and parcel of energy conservation. It will further help utility and policy makers gain better insights into energy consumption in residential buildings. 

The proposed study will establish tools and resource pertaining to energy consumption data sets. This will facilitate and promote research activities in energy disaggregation, energy data analysis, electricity grid modelling and appliance usage behaviour. Apart from that, energy consumption data sets  will be useful for policy makers in the energy sector.

The study will also contribute to the  understanding of challenges and possible strategies for energy conservation in residential buildings. Finally, the proposed study is expected to provide better theoretical understanding of NILM and its applicability in sustainable energy in residential buildings. }.
		\begin{minted}[baselinestretch=1,fontsize=\footnotesize]{tex}
		\newpage\thispagestyle{plain}~
		\clearpage
		\pagenumbering{arabic}
		\end{minted}
	\end{itemize}
\end{frame}

\begin{frame}[fragile]{Thesis with Latex}
	\begin{minted}[baselinestretch=1,fontsize=\footnotesize]{tex}
	documentclass[a4paper,12pt]{report} 
	
	\usepackage{lipsum}  %Generate dummy text (lorem ipsum) in your document
	\usepackage{algorithm2e}  %	
	\usepackage[round]{natbib} %for citation
	\usepackage{graphicx} % for images and figures
	
	\setcounter{secnumdepth}{3}
	\setcounter{tocdepth}{3}
	\pagenumbering{roman}
	
	\begin{document}
	\newcounter{rom}	
	\input{tex/tittle} % The tittle page
	\begin{center}
{\large \textbf{CERTIFICATION}} \\
\vspace{1in}
\end{center}
The undersigned certify that they have read and hereby recommend for acceptance by the University of Dodoma dissertation entitled \textbf{Maximizing LTE performance with MIMO Systems} in fulfillment of the requirements for the degree of \textbf{Master of Science in Telecommunications Engineering} of the Nelson Mandela African Institution of Science and Technology.

\vspace{1in}
\begin{center}
Approved by: 
\bigbreak
\noindent\begin{tabular}{ll}
	\makebox[2.5in]{\hrulefill} & \makebox[2.5in]{\hrulefill}\\
	Prof A.N Mvuma (First Supervisor) & Date\\[8ex]% adds space between the two sets of signatures
	\makebox[2.5in]{\hrulefill} & \makebox[2.5in]{\hrulefill}\\
	Dr. S Kaijage (Second Supervisor) & Date\\
\end{tabular}
\end{center}
\newpage
	
	\addtocounter{rom}{1}\setcounter{page}{2}~
	\newpage\thispagestyle{plain}\setcounter{page}{3}
	
	\tableofcontents % Create table of comments
	
	\end{document}
	
	\end{minted}
	
\end{frame}

\begin{frame}[fragile]{Thesis with Latex}
	\begin{minted}[baselinestretch=1,fontsize=\footnotesize]{tex}

	\listoffigures % list of figures
	\listoftables  % list of tables
	\listofalgorithms % list of algorithms
	
	\newpage\thispagestyle{plain}~
	\clearpage
	\pagenumbering{arabic}
	
	\section{Introduction}
Worldwide electrical energy consumption is constantly increasing especially in developing countries \citep{Monacchi2013}. According to United States Energy Information Administration report\footnote{http://www.eia.gov/todayinenergy/detail.cfm?id=14011}, \textit{the developing countries will account for 65\% of the global energy consumption by 2040}. 

\lipsum[1]

\subsection{Background}

Over the recently most part of the world has witnessed rapidly increasing to energy use in buildings (residential and commercial). Building contribute about 40\% of the total global energy \citep{Batra2014c}. Residential and commercial buildings consume approximately 60\% of the world’s electricity \footnote{The United Nation’s Environment Programme’s Sustainable Building and Climate Initiative (UNEP-SBCI)}. In U.S. A for example 74.9\% of all the electricity produced is used just to operate buildings \footnote{\href{http://www.eia.gov/todayinenergy/detail.cfm?id=14011}{United States Energy Information Administration report}} while in Africa, 56\% of the total electric power demand is from buildings \citep{Kitio2013}. Thus, energy saving in buildings will have significant impact on the reduction of overall energy consumption.

\lipsum[1]

\subsection{Problem Statement}

The key challenge to NILM problem is how to design efficient unsupervised NILM algorithm that can run in real-time using low-frequency sampling data. Several state-of-the-art NILM algorithms have been proposed using different approaches such as different variants of Hidden Markov Models (HMM) \citep{Kim2011,Parson2012,Kolter2012,Makonin2015}, Deep Neural Networks (DNN) \citep{Badayos2015,Paulo2016a}, Graph Signal Processing (GSP) \citep{Stankovic2014,Zhao2016a} and Combinatorial Optimization (CO) \citep{ReyesLua2015,Batra2013}. However, most of these algorithms suffer from high computational complexity which make them unstable for real-time applications, cannot be generalized across different buildings, requires lot of training data, their performance is limited to few numbers of appliances and are sensitive to noise and similar devices

\lipsum[1]
\subsection{Research Objectives}

\subsubsection{Main Objective}

The broad aim of this study is to develop NILM framework for sustainable residential buildings energy. The NILM framework will be well suited for the inherent characteristics of grids in Tanzania.

\subsubsection{Specific Objectives}
Specific Objectives are:
\begin{enumerate}
	\item To develop tools that will enable disaggregation research in developing countries.
	\item To develop innovative and real-time unsupervised NILM algorithms for sustainable energy consumption in residential buildings.
	\item To demonstrate and evaluate the potential of the proposed algorithm in (2) for sustainable
	energy consumption in residential buildings .
\end{enumerate}

\subsection{Research Questions}
This research is intended to answer the following questions:
\begin{enumerate}
	\item What tools can be developed to increase energy disaggregation research in developing countries?.
	\item How to design an efficient and real-time NILM algorithms that can be generalized across buildings  by taking into consideration developing countries characteristics?.
	\item Which and how innovative sustainable energy saving applications in residential buildings could be enabled with NILM algorithms in (2)?.
\end{enumerate}

\subsection{Significance of the Research}
The major contribution of the proposed study will be  novel unsupervised algorithm for energy disaggregation problem and its applicability in helping households to achieve quantifiable energy saving. The algorithm will provide real-time appliance specific information that will increase public awareness and make them be part and parcel of energy conservation. It will further help utility and policy makers gain better insights into energy consumption in residential buildings. 

The proposed study will establish tools and resource pertaining to energy consumption data sets. This will facilitate and promote research activities in energy disaggregation, energy data analysis, electricity grid modelling and appliance usage behaviour. Apart from that, energy consumption data sets  will be useful for policy makers in the energy sector.

The study will also contribute to the  understanding of challenges and possible strategies for energy conservation in residential buildings. Finally, the proposed study is expected to provide better theoretical understanding of NILM and its applicability in sustainable energy in residential buildings. 
	\chapter{Literature Review}
\lipsum[2]

	\chapter{Research Methodology}

\lipsum[2-4]

	\input{tex/Results}
	\chapter{Conclusion and Future Works}
\lipsum[2-4]

	
	\bibliographystyle{apa}
	\bibliography{bib/References}
	
	\end{document}
	
	\end{minted}
	
\end{frame}


\section{Presentation Slides}

\begin{frame}
	\centering
	\Huge{THANK YOU}
\end{frame}

\begin{frame}[t, allowframebreaks]

\frametitle{References}
\bibliographystyle{apa} %\bibliographystyle{apacite} 
\newpage
\bibliography{bib/References_NILM}
\end{frame} 

    \end{darkframes}


\end{document}
