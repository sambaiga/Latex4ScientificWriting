% !TeX document-id = {2421f23d-51b9-43a2-9dc6-52c6746474b3}
% !TeX TXS-program:compile = txs:///pdflatex/[--shell-escape]
\documentclass{bredelebeamer}    
% suppress navigation bar
\beamertemplatenavigationsymbolsempty

\mode<presentation>
{
  %\usetheme{bunsen}
  \setbeamercovered{transparent}
  \setbeamertemplate{items}[circle]
}

\beamertemplatenavigationsymbolsempty
\usepackage{color}
\definecolor{uipoppy}{RGB}{225, 64, 5}
\definecolor{uipaleblue}{RGB}{96,123,139}
\definecolor{uiblack}{RGB}{0, 0, 0}

% caption styling
\DeclareCaptionFont{uiblack}{\color{uiblack}}
\DeclareCaptionFont{uipoppy}{\color{uipoppy}}
\captionsetup{labelfont={uipoppy},textfont=uiblack}
\include{macros}
\usepackage{minted}


%%%%%%%%%%%%%%%%%%%%%%%%%%%%%%%%%%%%%%%%%%%%%%%%


%%%%%%%%%%%%%%%%%%%%%%%%%%%%%%%%%%%%%%%%%%%%%%%%

\title[Latex]{\textbf{LaTex for Scientific Writing: Day -1}}
% Titre du diaporama

%\subtitle{\textbf{PhD Student (NMAIST-CoCSE)} }
% Sous-titre optionnel

\institute[NMAIST]
{
	
	
	{\href{https://sambaiga.github.io/}{Anthony FAUSTINE}}. 
	\\[1.0cm]
	{\large
    The School of Computational and Communication Sciences and Engineering \\[1.0cm]
    NMAIST

	}
}







\date{\texttt{21 July 2017}}
% Optionnel. La date, généralement celle du jour de la conférence

\subject{Sujet de votre diaporama}
% C'est utilisé dans les métadonnes du PDF



\logo{
	\includegraphics[scale=0.1]{images/logo}
}
\hypersetup{
	pdfauthor = {Anthony Faustine: sambaiga@gmail.com},
	pdfsubject = {},
	pdfkeywords = {},
	pdfmoddate= {},
	pdfcreator = {}
}




%%%%%%%%%%%%%%%%%%%%%%%%%%%%%%%%%%%%%%%%%%%%%%%%%%%%%%%%%%%%%%%%%%%%%
\begin{document}

\begin{frame}
	\titlepage
\end{frame}




\AtBeginSection[] { \begin{frame} %
		<beamer> \frametitle{Outline} \setcounter{tocdepth}{2} \tableofcontents[currentsection, sectionstyle=show/shaded,hideallsubsections ] \end{frame} }



%++++++++++++++++++++++++++++++++++++++++++++++++++++

\begin{frame}{Presenter Bio}
\begin{itemize}
\item \textsf{PhD student at Nelson Mandela African Institution of Science and Technology,} 
\item \textbf{Research:}  \textsf{Applied machine learning and signal processing for computational sustainability}.
\begin{itemize}
	\item \textsf{Hybrid HMM-DNN for energy dis-aggregation problem}.
\end{itemize}	
\item \textsf{co-founder \href{https://pythontz.github.io/}{pythontz} [\texttt{https://pythontz.github.io}]}
\item \textsf{ass.Lecturer: \href{www.udom.ac.tz}{the University of Dodoma}}
\item \textsf{blog}: [\texttt{https://sambaiga.github.io}]
\end{itemize}
\end{frame}

%--------------------------------------------------------
\section{Graphics, Figures and Tables}

\begin{frame}[fragile]{Tables}{Creating Tables}
Use the tabular environment
\begin{minted}{tex}
\begin{tabular}[position]{column alignments}
...
\end{tabular}
\end{minted}

\begin{description}
	\item[postion] is optional (vertical position): [t] (top), [c]
	(center, this is default), [b] (bottom);
	\item [column alignments]: l (left-justified), c (center justified),
	and r (right-justified);
\end{description}
\end{frame}

\begin{frame}[fragile]{Tables}{Creating Tables}
The column data is separated by \textcolor{blue}{\&},row end is marked as \mintinline{tex}{\\} and \mintinline{tex}{\hline} draw a horizontal line.\\[0.5cm]
Consider the following simple table:\\

\centering
\begin{tabular}[c]{|l|l|}
\hline 
\textbf{Parameter} & \textbf{Value}\\
\hline 
Path loss (n) & 2.5\\
Model & Okumura-model\\
Cell-radius & $1km$\\
\hline
\end{tabular}

\end{frame}

\begin{frame}[fragile]{Tables}{Creating Tables}
Latex code for previous simple table\\
\begin{minted}{tex}
\centering
\begin{tabular}[t]{|l|l|}
\hline 
\textbf{Parameter} & \textbf{Value}\\
\hline 
Path loss (n) & 2.5\\
Model & Okumura-model\\
Cell-radius & $1km$\\
\hline
\end{tabular}
\end{minted}

\end{frame}



\begin{frame}[fragile]{Tables}{Creating Tables}
The \alert{\texttt{booktabs }} package  improve the quality  LaTeX tables.
\begin{itemize}
	\item The horizontal rules are called with \mintinline{tex}{\toprule}, \mintinline{tex}{\midrule} and \mintinline{tex}{\bottomrule} instead of \mintinline{tex}{\hline} command.
	\item The \mintinline{tex}{\cmidrule} is used for mid-rules that span specified columns.
	\item The content of the tables is filled in the same manner as before.
	\item To use this package first you need to add this code in preamble. \mintinline{tex}{\usepackage{booktabs}}
\end{itemize}
\end{frame}

\begin{frame}[fragile]{Tables}{Creating Tables}
\alert{Example:} 

\centering
\begin{tabular}[t]{ll}
\toprule
\textbf{Parameter} & \textbf{Value}\\
\midrule
Path loss (n) & 2.5\\
Model & Okumura-model\\
Cell-radius & $1km$\\
\bottomrule
\end{tabular}

\end{frame}

\begin{frame}[fragile]{Tables}{Creating Tables}
 code:\
\begin{minted}{tex}
\centering
\begin{tabular}[t]{ll}
\toprule
\textbf{Parameter} & \textbf{Value}\\
\midrule
Path loss (n) & 2.5\\
Model & Okumura-model\\
Cell-radius & $1km$\\
\bottomrule
\end{tabular}
\end{minted}
\end{frame}



\begin{frame}[fragile]{Tables}{Creating Tables}
To draw multicolumn table like this one:\\[0.5cm]
\begin{center}
\begin{tabular}{llr}
\toprule
\multicolumn{2}{c}{Name} \\
\cmidrule(lr){1-2}
First name	& Last Name	& Grade \\
\midrule
John 		    & Doe	  		& $7.5$ \\
Richard 	  & Miles		  & $2$ \\
\bottomrule
\end{tabular}
\end{center}
\end{frame}

\begin{frame}[fragile]{Tables}{Creating Tables}
Use the following command: \mintinline{tex}{\multicolumn{n}{alignment}{item}}
\begin{description}
\item [n]: is the number of columns to be spanned.
\item [alignemnt]: is one of the \mintinline{tex}{l, r and c}.
\item [item]: is the content.
\end{description}
\alert{Example:}
\begin{minted}[numbersep=5pt]{tex}

\multicolumn{2}{c}{Name} 

\end{minted}

\end{frame}

\begin{frame}[fragile]{Tables}{Creating Tables}

\begin{minted}{tex}
\begin{tabular}{llr}
\toprule
\multicolumn{2}{c}{Name} \\
\cmidrule(r){1-2}
First name & Last Name	& Grade \\
\midrule
John  & Doe	& $7.5$ \\
Richard   & Miles  & $2$ \\
\bottomrule
\end{tabular}
\end{minted}

\end{frame}

\begin{frame}[fragile]{Tables}{Floating Tables}
Latex	provides the \texttt{table}  environments for typesetting floating tables.
\begin{itemize}
\item A table environment is set up as follows:
\begin{minted}{tex}
\begin{table}
\caption{title}
\label{tab:xxx}
%Place the table here
\end{table}
\end{minted}
\end{itemize}
\end{frame}

\begin{frame}[fragile]{Tables}{Floating Tables}
\begin{description}
\item [\mintinline{tex}{\caption}] command is optional and used to set table tittle.
\item [\mintinline{tex}{\label}] command is also optional and is used to reference
the table’s number.
\end{description}
\alert{Example:} To produce the following table
\begin{table}
\caption{Simulation Parameters}
\label{tab:model_parameter}
\begin{tabular}[t]{ll}
\toprule
\textbf{Parameter} & \textbf{Value}\\
\midrule
Path loss (n) & 2.5\\
Model & Okumura-model\\
Cell-radius & $1km$\\
\bottomrule
\end{tabular}
\end{table}

\end{frame}


\begin{frame}[fragile]{Tables}{Floating Tables}
\footnotesize
\begin{minted}{tex}
\begin{table}
\caption{Simulation Parameters}
\label{tab:model_parameter}
\begin{tabular}[t]{ll}
\toprule
\textbf{Parameter} & \textbf{Value}\\
\midrule
Path loss (n) & 2.5\\
Model & Okumura-model\\
Cell-radius & $1km$\\
\bottomrule
\end{tabular}
\end{table}
\end{minted}

\alert{Guide to Making nice Table}: \href{https://www.inf.ethz.ch/personal/markusp/teaching/guides/guide-tables.pdf}{click here}
\end{frame}

\begin{frame}[fragile]{Activity}
\begin{center}
	{\Large \textbf{Activity 1}}
\end{center}

\end{frame}

\begin{frame}[fragile]{Graphics}{Include Graphics}
The easiest way to include images in your document is to use
the \texttt{graphicx} package.\\
Load the package graphicx:\mintinline{tex}{ \usepackage{graphicx}}
\begin{itemize}
	\item The image format available depend on what you’re using to compile.
	\item If you’re compiling using pdflatex (recommended), then you
	can use jpg, png, pdf, or eps files.
	\item Place the file in the same directory as your tex file, and use the
	\mintinline{tex}{\includegraphics[key-values]{imagefile}} command.
\end{itemize}
\end{frame}

\begin{frame}[fragile]{Graphics}{Include Graphics}
\alert{Example:} 
\mintinline{tex}{\includegraphics[scale=0.2]{images/bulb}}
\begin{center}
\includegraphics[scale=0.2]{images/bulb}
\end{center}

\end{frame}

\begin{frame}[fragile]{Graphics}{Include Graphics}
The image can be scaled to a specified height and/or width as follows:
\mintinline{tex}{\includegraphics[height=2in,width=1in]{images/bulb}}
\begin{center}
\includegraphics[height=1.5in,width=1in]{images/bulb}
\end{center}
\end{frame}

\begin{frame}[fragile]{Graphics}{Floating images}
Use \texttt{figure} environment:
\begin{center}
\begin{minted}{tex}
\begin{figure}
\includegraphics{file}
\caption{title }
\label{fig:xxx}
\end{figure}
\end{minted}
\end{center}
\end{frame}



\begin{frame}[fragile]{Graphics}{Floating images}
\begin{figure}
\includegraphics[scale=0.15]{images/bulb}
\caption{Green Bulb} 
\label{fig:bulb}
\end{figure}
To print the list of figures and tables use \mintinline{tex}{ \listoffigures} and \mintinline{tex}{\listoftables}  respectively.\\		

\end{frame}

\begin{frame}[fragile]{Activity}
\begin{center}
{\Large \textbf{Activity 2}}
\end{center}

\end{frame}


\begin{frame}[fragile]{Notes with todonotes}
The \mintinline{tex}{\todo} command from the \texttt{todonotes} package is great for leaving notes to yourself and your collaborators.\\

Include this package in preamble: \mintinline{tex}{\usepackage[colorinlistoftodos]{todonotes}}\\[0.5cm]

\alert{Example:}\\
\mintinline{tex}{\todo{Plain todonotes.}}  \\[0.50cm]
\emph{Footnotes}:The command you need is \mintinline{tex}{\footnotes{An example footnote}} \footnote{An example footnote}

\end{frame}

\begin{frame}[fragile]{Notes with todonotes}
\begin{itemize}
\item Only inline notes are supported with
beamer, but margin notes are supported for normal
documents.
\item There is also a handy \mintinline{tex}{\listoftodos} command.
\begin{itemize}
	\item To use this load: \mintinline{tex}{\usepackage[colorlinks]{hyperref}} before \mintinline{tex}{\usepackage[colorinlistoftodos]{todonotes}}
\end{itemize}
\end{itemize}

\end{frame}

\begin{frame}[fragile]{Activity}
\begin{center}
{\Large \textbf{Activity 3}}
\end{center}

\end{frame}


\section{Bibliography}

\begin{frame}{Bibliography}
To manage and include references in a LATEX document use \textsc{BibTeX}.\\
\begin{description}[long text]
	\item[BibTex]: a bibliographic tool that is used with LaTeX to help organize the user's references and create a bibliography.
\end{description}
\begin{itemize}
	\item A BibTeX user creates a bibliography file with \mintinline{tex}{.bib} extension.
	\item The \mintinline{tex}{.bib} file is called a BibTEX database.
	\item Each entry in the \mintinline{tex}{.bib} file is formatted with a certain structure and is given a \alert{"key"} by which the author can refer to it in the source file.
\end{itemize}
\end{frame}

\begin{frame}[fragile]{Bibliography}{The BibTEX Format}
The generic form of a BibTEX entry is
\begin{center}
\begin{minted}{tex}
@type{key, field1 = ‘‘ ’’ or {} or none,
field2 = ‘‘ ’’ or {} or none, 
... 
fieldn = ‘‘ ’’ or {} or none 
}
\end{minted}
\end{center}
\end{frame}
\begin{frame}[fragile]{Bibliography}{The BibTEX Format}
The generic form of a BibTEX entry is
\begin{center}
	\footnotesize
\begin{minted}{tex}
@INPROCEEDINGS{Pantic2006, 
author={M. Pantic and R. Zwitserloot}, 
booktitle={Proceedings. Frontiers in Education. 36th Annual Conference}, 
title={Active Learning of Introductory Machine Learning}, 
year={2006}, 
pages={1-6},  
doi={10.1109/FIE.2006.322738}, 
ISSN={0190-5848}, 
month={Oct},}
\end{minted}
\end{center}
\end{frame}

\begin{frame}[fragile]{Bibliography}{The BibTEX Format}
\alert{Example:}
\flushleft{
\begin{minted}{tex}
@article{Gettys90,
author = {Jim Gettys and Phil Karlton and Scott McGregor},
title = {The {X} Window System, Version 11},
journal = {Software Practice and Experience},
volume = {20},
number = {S2},
year = {1990},
abstract = {A technical overview of the X11 functionality.  This is an update
of the X10 TOG paper by Scheifler \& Gettys.}
}
\end{minted}
}
\end{frame}

\begin{frame}{Bibliography}{Export .bib file from Mendeley}
To export BiBTex:
\begin{enumerate}
	\item Open Mendeley, and within "My Library" found on the left, select references that you would like to texport to BibTeX.
	\item In the drop-down menu in the toolbar at the top of the screen, click "File $\rightarrow$ Export.
	\item In the dropdown list of filetypes chose "BibTeX (*.bib)" and save to the same location as the LaTeX file.
\end{enumerate}

\alert{More on Mendeley and LaTeX}:\href{http://libguides.mit.edu/c.php?g=176186\&p=1159535\#3}{here}.
\end{frame}


\begin{frame}[<+->]{Bibliography}{Auto-syncing from Mendeley to BibTeX}
Mendeley has the built-in capability to auto sync a BibTeX file when changes have been made to your Mendeley library.\\
To set up the Mendeley auto sync:
\begin{enumerate}
	\item Go to Mendeley Desktop preferences.
	\item Select the BibTeX tab. 
	\item Select the box labeled “Enable BibTeX syncing” and select the BiBTeX file option you prefer.
	\item Select the location where you want the generated .bib file(s) to be stored (this should be the same location as your LaTeX file(s)).
\end{enumerate}
\end{frame}

\begin{frame}[fragile]{Activity}
\begin{center}
{\Large \textbf{Activity 4}}
\end{center}

\end{frame}



\begin{frame}{Bibliography}{Using BibTEX in your LATEX Document}
To use .bib file in  latex document:
\begin{itemize}
	\item We can use \mintinline{tex}{natbib} packages with \mintinline{tex}{\citet and \citep} commands $\Rightarrow$ \href{http://merkel.texture.rocks/Latex/natbib.php}{Reference sheet for natbib usage}.
	\item Load with \mint{tex}{\usepackage[options]{natbib}} See list of options at the end of \href{http://merkel.texture.rocks/Latex/natbib.php}{Reference sheet for natbib usage}.
	\item Example: \mintinline{tex}{\usepackage[round]{natbib}}
	\item Include .bib file at the end of document with \mintinline{tex}{\bibliography{bib file}} and specify a bibliographic styles \mintinline{tex}{\bibliographystyle{stylename}}.
	
\end{itemize}
\end{frame}


\begin{frame}[fragile]{Bibliography}{The BibTEX Format}
\alert{Example:}

\begin{minted}{tex}


\bibliographystyle{apa} %\bibliographystyle{apacite} 
\newpage
\bibliography{bib/References_NILM}
\end{minted}
\end{frame}


\begin{frame}[fragile]{Bibliography}{The BibTEX Format}
\alert{Example:} 
\begin{itemize}
	\item According to \citet{Barker2015} \ldots $\Rightarrow$ \mintinline{tex}{According to \cite{Alcala2015} \ldots}
	\item \ldots energy is important \citep{Barker2015}  $\Rightarrow$ \mintinline{tex}{\ldots energy is important \citep{Barker2015}}
\end{itemize}	
\end{frame}

\begin{frame}[fragile]{Activity}
\begin{center}
{\Large \textbf{Activity 5}}
\end{center}

\end{frame}

\section{Proposal, Thesis and Joural paper with Latex}
\begin{frame}{Proposal, Thesis and Joural paper with Latex}
\alert{Folder structure}: create a new folder (your project directory). 
\begin{itemize}
	\item  Add some additional folders within this folder:
	\begin{description}
		\item [fig] $\Rightarrow$ will contain all images.
		\item [tex] $\Rightarrow$ will contain \textsc{.tex} file.
		\item [bib] $\Rightarrow$ will contain bibliography files.
	\end{description}
	\item This will help you keep the overview about your files.
\end{itemize}
Latex template make life easier.
\end{frame}

\begin{frame}[fragile]{Activity}
\begin{center}
	{\Large \textbf{Activity 6}}
\end{center}

\end{frame}

\section{Presentation Slides: LaTeX Beamer}

\begin{frame}[fragile]{LaTeX Beamer}
The beamer class is a LaTeX class that allows you to create a
beamer presentation and slides.
\begin{block}{The basic steps to create a beamer presentation }
	\begin{itemize}
		\item Specify beamer as document class instead of article.
		\item Structure your LaTeX text using section and subsection
		commands.
		\item Place the text of the individual slides inside frame commands.
	\end{itemize}
\end{block}

\end{frame}

\begin{frame}[fragile]{Frames}

Each beamer project is made up of a series of frames defined by \mintinline{tex}{\begin{frame}..\end{frame}} environment.
\begin{itemize}
\item Each frame produces one or more slides depending on the slide overlays.
\end{itemize}
\begin{description}
\item[The title page frame:]  simply displays a title page
\begin{center}
\mint{tex}{\begin{frame} \titlepage \end{frame}}
\end{center}
\item [The table of contents: ] dynamically creates a table of contents
based on the sections and subsections.
\end{description}
\begin{minted}{tex}
\begin{frame}{Outline}\tableofcontents \end{frame}
\end{minted}
\end{frame}


\begin{frame}[fragile]{Frame}
Basic Frame:
\mint{tex}{\begin{frame}{Tittle}
content...
\end{frame}}

\end{frame}

\begin{frame}[fragile]{Sections and Subsections}
Presentations are divided into sections, subsections, and
sub-subsections.
\end{frame}

\begin{frame}[fragile]{Activity}
\begin{center}
{\Large \textbf{Activity 7}}
\end{center}

\end{frame}

\begin{frame}[fragile]{Structuring a Frame}
Beamer provides many ways to structure your frames so they appear
well organized and are easy for the audience to follow.
\begin{itemize}
\item Columns
\item Blocks
\end{itemize}
\end{frame}

\begin{frame}[fragile]{Structuring a Frame}\framesubtitle{Coulmns}
The column environment is called as shown below:
\begin{center}
\begin{minted}{tex}
\begin{columns}
\column{.xx\textwidth}
First column text/graphics
\column{.xx\textwidth}
Second column text/graphics
\end{columns}
\end{minted}
\end{center}
\end{frame}

\begin{frame}[fragile]{Structuring a Frame}\framesubtitle{Coulmns}
Here is a simple example:
\begin{minted}{tex}
\begin{columns}
\column{.5\textwidth}
Column Number 1
\column{.5\textwidth}
Column Number 2
\end{columns}
\end{minted}
\alert{Which gives us}:\\
\begin{columns}
\column{.5\textwidth}
Column Number 1
\column{.5\textwidth}
Column Number 2
\end{columns}	
\end{frame}

\begin{frame}[fragile]{Structuring a Frame}\framesubtitle{Blocks}
Blocks can be used to separate a specific section of text or graphics
from the rest of the frame.

\begin{center}
\begin{minted}{tex}
\begin{block}{Block Tittle}
content...
\end{block}
\end{minted}
\end{center}
\alert{Which gives us}:\\
\begin{block}{Block Tittle}
content...
\end{block}
\end{frame}


\begin{frame}[fragile]{Structuring a Frame}\framesubtitle{Blocks}
Other block environments are also available:\\
\textbf{Theorem block}	
\begin{center}
\begin{minted}{tex}
\begin{theorem}
$x^2 + y^2 = 1 \rightarrow$ Circle with $r = 1$
\end{theorem}
\end{minted}
\end{center}
\alert{Which gives us}:\\
\begin{theorem}
$x^2 + y^2 = 1 \rightarrow$ Circle with $r = 1$
\end{theorem}
\end{frame}

\begin{frame}[fragile]{Structuring a Frame}\framesubtitle{Blocks}
Other block environments are also available:\\
\textbf{Example block}	
\begin{center}
\begin{minted}{tex}
\begin{example}
This is $\ldots$
\end{example}
\end{minted}
\end{center}
\alert{Which gives us}:\\
\begin{example}
This is $\ldots$
\end{example}
\end{frame}

\begin{frame}[fragile]{Structuring a Frame}\framesubtitle{Blocks}
Other block environments are also available:\\
\textbf{Alert block}	
\begin{center}
\begin{minted}{tex}
\begin{alertblock}{Title}
This is $\ldots$
\end{alertblock}
\end{minted}
\end{center}
\alert{Which gives us}:\\
\begin{alertblock}{Title}
This is $\ldots$
\end{alertblock}
\end{frame}

\begin{frame}[fragile]{Structuring a Frame}\framesubtitle{Blocks}
Other block environments are also available:\\
\textbf{Lemma block}	
\begin{center}
\begin{minted}{tex}
\begin{lemma}
$x^2 + y^2 = 1$ 
\end{lemma}
\end{minted}
\end{center}
\alert{Which gives us}:\\
\begin{lemma}
$x^2 + y^2 = 1 $
\end{lemma}
\end{frame}

\begin{frame}[fragile]{Structuring a Frame}\framesubtitle{Blocks}
Other block environments are also available:\\
\textbf{Corollary block}	
\begin{center}
\begin{minted}{tex}
\begin{corollary}
This proof
\end{corollary}
\end{minted}
\end{center}
\alert{Which gives us}:\\
\begin{corollary}
This proof
\end{corollary}
\end{frame}

\begin{frame}[fragile]{Structuring a Frame}\framesubtitle{Blocks}
Other block environments are also available:\\
\textbf{Proof block}	
\begin{center}
\begin{minted}{tex}
\begin{proof}
This proof
\end{proof}
\end{minted}
\end{center}
\alert{Which gives us}:\\
\begin{proof}
This proof
\end{proof}
\end{frame}

\begin{frame}[fragile]{Structuring a Frame}\framesubtitle{Columns and Blocks}

We can combine columns and blocks to make a much cleaner looking
presentation.
\begin{center}
\begin{minted}{tex}
\begin{columns}[t]
\column{.5\textwidth}
\begin{block}{Column 1 Header}
Column 1 Body Text
\end{block}
\column{.5\textwidth}
\begin{block}{Column 2 Header}
Column 2 Body Text
\end{block}
\end{columns}
\end{minted}
\end{center}


\end{frame}

\begin{frame}[fragile]{Structuring a Frame}\framesubtitle{Columns and Blocks}
\alert{Which gives us}:\\
\begin{columns}[t]
\column{.5\textwidth}
\begin{block}{Column 1 Header}
Column 1 Body Text
\end{block}
\column{.5\textwidth}
\begin{block}{Column 2 Header}
Column 2 Body Text
\end{block}
\end{columns}


\textbf{Note}. Notice that the \mintinline{tex}{[t]} argument to the columns command top-aligned our blocks so they are vertically even as opposed to vertically centered on the slide.
\end{frame}


\begin{frame}[fragile]{Overlay}
In Beamer, overlays control the order in which parts of the frame
appear.
\begin{itemize}
\item An easy way to implement an overlay is to place the \mintinline{tex}{\pause} command between the parts you want to show up separately.
\end{itemize}


\alert{Example}
\begin{center}
\begin{minted}{tex}
\textbf{Step1:} Compute the maximal value of $x^2$.
\pause
\textbf{Step2:} Compute the maximal value of $y^2$.
\pause
\textbf{Step3:} Add $x^2$ and $y^2$.
\end{minted}
\end{center}
\end{frame}


\begin{frame}[fragile]{Overlay}
\alert{Which results into:}\\

\textbf{Step1:} Compute the maximal value of $x^2$.\\
\pause
\textbf{Step2:} Compute the maximal value of $y^2$.\\
\pause
\textbf{Step3:} Add $x^2$ and $y^2$.
\end{frame}


\begin{frame}[fragile]{Overlay}
\alert{Which results into:}\\

\textbf{Step1:} Compute the maximal value of $x^2$.\\
\pause
\textbf{Step2:} Compute the maximal value of $y^2$.\\
\pause
\textbf{Step3:} Add $x^2$ and $y^2$.
\end{frame}

\begin{frame}[fragile]{Overlay Specification}
Overlay specifications are given in pointed brackets (<,>) and
indicate which slide the corresponding information should appear
on.
\begin{itemize}
\item The specification \mintinline{tex}{<1->} means “display from slide 1 on.” 
\item \mintinline{tex}{<1-3>} means “display from slide 1 to slide 3.” 
\item  \mintinline{tex}{<2>} means “display  slide 2.” 
\item \mintinline{tex}{<-3,5-6,8->} means “display on all slides except slides 4 and 7.”
\end{itemize}
\textbf{Note}: If you want each item of a list to appear in order, use the \mintinline{tex}{[<+->] option}\\
E.g. \mintinline{tex}{\begin{frame}[<+->]{title} ..... \end{frame}}
\end{frame}

\begin{frame}[fragile]{Overlay } \framesubtitle{Timing Specifications With Alert}
\alert{Consider:}

\begin{columns}
\begin{column}{0.5\textwidth}
\begin{itemize}[<+-| alert@+>]
\item Item Aaay
\item Item Bee
\item Item See
\item Item Dee
\end{itemize}
\end{column}
\pause
\hspace{5pt}\vrule\hspace{5pt}%
\begin{column}{0.5\textwidth}		
\begin{minted}{tex}
\begin{itemize}[<+-| alert@+>]
\item Item Aaay
\item Item Bee
\item Item See
\item Item Dee
\end{itemize}
\end{minted}
\end{column}
\end{columns}		
\end{frame}




\begin{frame}[fragile]{Activity}
\begin{center}
{\Large \textbf{Activity 8}}
\end{center}

\end{frame}
%-------------------------------------------------------------
\begin{frame}
\centering
\emph{THANK YOU}
\end{frame}

\begin{frame}[t, allowframebreaks]

\frametitle{References}
\bibliographystyle{apa} %\bibliographystyle{apacite} 
\bibliography{bib/References_NILM}
\end{frame} 


\end{document}

