% !TeX document-id = {2421f23d-51b9-43a2-9dc6-52c6746474b3}
% !TeX TXS-program:compile = txs:///pdflatex/[--shell-escape]
\documentclass{bredelebeamer}    
% suppress navigation bar
\beamertemplatenavigationsymbolsempty

\mode<presentation>
{
  %\usetheme{bunsen}
  \setbeamercovered{transparent}
  \setbeamertemplate{items}[circle]
}

\beamertemplatenavigationsymbolsempty
\usepackage{color}
\definecolor{uipoppy}{RGB}{225, 64, 5}
\definecolor{uipaleblue}{RGB}{96,123,139}
\definecolor{uiblack}{RGB}{0, 0, 0}

% caption styling
\DeclareCaptionFont{uiblack}{\color{uiblack}}
\DeclareCaptionFont{uipoppy}{\color{uipoppy}}
\captionsetup{labelfont={uipoppy},textfont=uiblack}
\include{macros}
\usepackage{minted}


%%%%%%%%%%%%%%%%%%%%%%%%%%%%%%%%%%%%%%%%%%%%%%%%


%%%%%%%%%%%%%%%%%%%%%%%%%%%%%%%%%%%%%%%%%%%%%%%%

\title[Latex]{\textbf{LaTex for Scientific Writing: Day -1}}
% Titre du diaporama

%\subtitle{\textbf{PhD Student (NMAIST-CoCSE)} }
% Sous-titre optionnel

\institute[NMAIST]
{
	
	
	{\href{https://sambaiga.github.io/}{Anthony FAUSTINE}}. 
	\\[1.0cm]
	{\large
    The School of Computational and Communication Sciences and Engineering \\[1.0cm]
    NMAIST

	}
}







\date{\texttt{21 July 2017}}
% Optionnel. La date, généralement celle du jour de la conférence

\subject{Sujet de votre diaporama}
% C'est utilisé dans les métadonnes du PDF



\logo{
	\includegraphics[scale=0.1]{images/logo}
}
\hypersetup{
	pdfauthor = {Anthony Faustine: sambaiga@gmail.com},
	pdfsubject = {},
	pdfkeywords = {},
	pdfmoddate= {},
	pdfcreator = {}
}




%%%%%%%%%%%%%%%%%%%%%%%%%%%%%%%%%%%%%%%%%%%%%%%%%%%%%%%%%%%%%%%%%%%%%
\begin{document}

\begin{frame}
	\titlepage
\end{frame}




\AtBeginSection[] { \begin{frame} %
		<beamer> \frametitle{Outline} \setcounter{tocdepth}{2} \tableofcontents[currentsection, sectionstyle=show/shaded,hideallsubsections ] \end{frame} }



%++++++++++++++++++++++++++++++++++++++++++++++++++++
\section{Introduction}
\begin{frame}{Presenter Bio}
\begin{itemize}
\item \textsf{PhD student at Nelson Mandela African Institution of Science and Technology,} 
\item \textbf{Research:}  \textsf{Applied machine learning and signal processing for computational sustainability}.
\begin{itemize}
	\item \textsf{Hybrid HMM-DNN for energy dis-aggregation problem}.
\end{itemize}	
\item \textsf{co-founder \href{https://pythontz.github.io/}{pythontz} [\texttt{https://pythontz.github.io}]}
\item \textsf{ass.Lecturer: \href{www.udom.ac.tz}{the University of Dodoma}}
\item \textsf{blog}: [\texttt{https://sambaiga.github.io}]
\end{itemize}
\end{frame}


  \begin{frame}[<+->]{What is Latex}
  	A very powerful text (markup) processing system designed to produce quality typeset
  	documents.
  	\begin{itemize}
  		\item It is based on the \alert{TEX}: A typesetting system $\Rightarrow$ designed and created by Donald Knuth in 1978
  		\item \alert{LaTeX} is a user-friendly extension of TeX $\Rightarrow$  a slightly higher-level language built on top of TEX.
  		\begin{itemize}
  			\item TeX and LaTeX $\Rightarrow$ assembly language and C
  		\end{itemize}
  		
  	\end{itemize}
  \end{frame} 

 \begin{frame}[<+->]{Why Latex}
   	
   	\begin{exampleblock}{\textbf {LATEX strength:}}
   		\begin{itemize}
   			\item Less focus on formatting and more on content.
   			\item It makes beautiful documents.
   			\item Superior and flexible equation presentation.
   			\item It was created by scientists, for scientists $\Rightarrow$ A large and active community.
   			\item Good for collaborative writing.
   			\item Fast, stable, extensible, and free (distribution dependent).
   		\end{itemize}
   	\end{exampleblock}
   \end{frame}
   

   \begin{frame}{How does it work?}
   	
   	
   	\begin{itemize}
   		\item You write your document in plain text with commands that describe its structure and meaning.
   		\item  The latex program processes your text and commands to produce a beautifully formatted document.
   	\end{itemize}
   	\pause
   	\begin{figure}
   		\includegraphics[scale=0.45]{images/latexhowthis}
   	\end{figure}
   \end{frame}


   \begin{frame}[<+->]{Fact about Latex}
   	
   	The most important fact about Latex:
   	\begin{multicols}{2}
   	\begin{figure}
   	\includegraphics[width=0.45\textwidth]{images/watch}
   	\end{figure}
   	\columnbreak
   	\includegraphics[width=0.45\textwidth]{images/google}
  \end{multicols}
   \end{frame}
  
  \begin{frame}[<+->]{Installation}
  \textbf{First you need a TEX Distribution:} contains all the software that you need to create a LATEX document.
  \begin{itemize}
  	\item \href{http://miktex.org/}{MiKTeX}: A free TeX distribution for Windows systems.
  	\item \href{http://www.tug.org/mactex/}{MacTeX}: A  free TeX distribution for Mac.
  	\item \href{https://www.tug.org/texlive/}{TeXLive}: A  free TeX distribution for for most flavors of Unix and windows.
  	\item  For more Latex info: https://www.latex-project.org/
  \end{itemize}
  
\end{frame}


\begin{frame}{Installation}
\textbf{You also need a text editor:} To create a LATEX source file
\begin{itemize}
\item \href{http://www.xm1math.net/texmaker/}{Texmaker}.
\item \href{http://www.texstudio.org/}{TexStudio}.
\item \href{https://www.sublimetext.com/3}{Sublime Text 3}.
\end{itemize}
We will use TexStudio with MiKTex
\begin{itemize}
	\item Download \href{http://www.texstudio.org/}{TexStudio} for your distribution
	\item Install TexStudio when MiKTeX installation is completed.
	\item TexStudio will automatically configure the settings for you.
\end{itemize}
 The installation of LaTeX is now complete.

\end{frame}

\begin{frame}{Online versions}
Popular online versions
\begin{itemize}
	\item Overleaf [https://www.overleaf.com/].
	\item Sharelatex [https://www.sharelatex.com/].
	\item Papeeria [https://papeeria.com/].
	\item Authorea [https://www.authorea.com/].
\end{itemize}
\end{frame}  



%-----------------------------------------------------------
\section{Latex Command}

\begin{frame}[fragile]{Activity}
\begin{center}
	{\Large \textbf{Activity 1}}
\end{center}

\end{frame}

\begin{frame}{Commands}
A LATEX document is mainly defined through commands. \\[1.0cm] 
Commands are case sensitive, and take one of the following two formats:
\begin{itemize}
\item They start with a backslash \alert{\textbackslash}  and then a name consisting of letters only.
\item Some commands need an argument, which has to be given between curly braces \{  \}.
\item Some commands support optional parameters, which
are added in square brackets [ ].
\end{itemize}
\end{frame}


\begin{frame}[fragile]{Commands}
\framesubtitle{Arguments and Options}
Many commands require a single argument, and some commands require even multiple arguments.
\begin{itemize}
	\item Some commands can have several options.
\end{itemize}
\pause
\alert{Example:}
\begin{minted}{tex}
\section{Introduction} % single argument
\usepackage{amsmath, amssymb} % multiple arguments
\documentclass[a4paper,11pt]{article} % several options
\usepackage[final]{microtype}  % single options
\end{minted}
\end{frame}


\begin{frame}[fragile]{Commands}
\framesubtitle{Environment}
An environment is be marked by, \mintinline{tex}{\begin{environment} ... \end{environment}}.

\begin{itemize}
	\item These initiate and exit an environment.
	\item The type of environment is applied to everything between the begin and end commands.
\end{itemize}\pause
\alert{Example:}
\begin{minted}{tex}
\begin{document}
content...           % document environment
\end{document}
\end{minted}
\end{frame}


%--------------------------------------------------------
\section{Document Structure}


\begin{frame}[fragile]{Document Structure}
Every LaTeX document has the following form:
\\[0.50cm]
\begin{minted}{tex}
\documentclass[options]{class name}

%Preamble

\begin{document}

%Body

\end{document}
\end{minted}

\end{frame}



\begin{frame}[fragile]{Document Class}

The command \mintinline{tex}{\documentclass[options]{class name}}  specify type of document you wants to create.
\begin{itemize}
	\item \alert{class name}: specifies the type of document to be created.
	\item \alert{options parameter}:customises the behaviour of the document class.
\end{itemize}

\pause
\alert{	Example:} 
\mint{latex}{\documentclass[11pt,a4paper]{article}}	
\end{frame}

\begin{frame}{Document Class}
Lists of the
document classes type.\\[1.0cm]
\centering
\scalebox{0.8}{
	\begin{tabular}{lp{12cm}}  
		\toprule
		\textbf{Class}    & \textbf{Description} \\
		\midrule
		article  & For articles in scientific journals, presentations, short reports, program documentation, invitations etc. \\
		report & For longer reports containing several chapters, small books, thesis etc.\\
		book & For real books.\\
		letter & For writing letters.\\
		beamer & For writing presentation\\
		exam  & For writing exams.\\
		\bottomrule
	\end{tabular}%
}
\label{tab:class1}%
\vspace{1.5in}
\end{frame}



\begin{frame}{Document Class: Options}
The
document classes options.
\centering
\scalebox{0.8}{
	\begin{tabular}{lp{10cm}}  
		\toprule
		\textbf{Options}    & \textbf{Description} \\
		\midrule
		10pt, 11pt, 12pt  & Sets the size of the main font in the document. Default is 10pt. \\
		a4paper,letterpaper.. & Defines the paper size.The default size is
		letterpaper.Besides that, a5paper, b5paper, executivepaper,
		and legalpaper can be specified.\\
		twocolumn & Instructs LaTeX to typeset the document in two columns instead of one.\\
		twoside, oneside & For writing letters.\\
		landscape & Changes the layout of the document to print in landscape mode.\\
		titlepage, notitlepage & Specifies whether a new page should be started
		after the document title or not. The article class does not start a
		new page by default, while report and book do.\\
		\bottomrule
	\end{tabular}%
}
\label{tab:option}%

\end{frame}

\begin{frame}[fragile]{Activity}
\begin{center}
	{\Large \textbf{Activity 2}}
\end{center}

\end{frame}

\begin{frame}[fragile]{The Preamble}
The preamble is where you \alert{define the style} of your document
and \alert{load any packages} you need to use.
\begin{minted}{tex}
\documentclass[options]{class name}

%Preamble

\begin{document}
\end{minted}

\begin{itemize}
	\item It normally contains commands, variables or other things needed that affect the entire document.
	\item Load needed packages along with any options for those packages.
\end{itemize}

\end{frame}


\begin{frame}[fragile]{The Preamble}
The preamble is also used to load any other options or information
that isn't necessarily a part of the document's content such as:
\begin{itemize}
\item Setting lengths of spaces before/after paragraphs, line height, etc
\item Specifying author/title/date, etc. (important if you will be making a title
page).
\end{itemize}

\end{frame}


\begin{frame}[fragile]{The Preamble}
\framesubtitle{Document Tittle}
There are two steps to give your document a title.
\begin{itemize}
	\item Tell LaTeX what to put in the title, and tell LaTeX to typeset the title.
	\item 	To specify title use the following commands in preamble:
	\mintinline{tex}{\title{...}, \author{...}, \date{..}}.
	\item  To display the title, use \mintinline{tex}{\maketitle} just after \mintinline{tex}{\begin{document}}.
	\end{itemize}
	\alert{Example:}
	\begin{minted}{tex}
	\title{Scintific Writing using LaTeX}
	\author{M.~Chuwa \and S.~Nyondo}
	\date{\today}
	\end{minted}
\end{frame}



\begin{frame}[fragile]{The Preamble: Packages}
Packages extend the basic LATEX commands.
\begin{itemize}
	\item To use packages, include the following  command:\mint{tex}|\usepackage[options]{package}|
	\item This command goes into the preamble of the document.
\end{itemize}
\alert{Example:}
\begin{minted}{tex}
%To set margin
\usepackage[top=2in,bottom=1in,left=1in,right=1in]{geometry}  
\usepackage{microtype} %improves the spacing between words and letters
\usepackage{amsmath} %introduces several improvements for math environments
\usepackage{graphicx} % for inserting image in latex document
\end{minted}
\end{frame}

\begin{frame}[fragile]{Activity}
\begin{center}
	{\Large \textbf{Activity 3}}
\end{center}

\end{frame}



\begin{frame}{The Body of the Document}
After the preamble comes the \alert{body}.
\begin{itemize}
	\item Starts with \mintinline{tex}{\begin{document}} and ends with \mintinline{tex}{\end{document}}
	\item This is where you fill in the actual content of your document.
	\item Contains all text, fgures, tables, etc.
\end{itemize}
\end{frame}


\begin{frame}{The Body of the Document}
You can organize your document using the following commands.
\begin{figure}
	\includegraphics[scale=0.5]{images/structure}
\end{figure}
\begin{itemize}
	\footnotesize
	\item Your PDF output will include these sections as bookmarks.
	\item The above commands have a *-version and using these results in no number
	and no entry in the table of contents.
	\item Example: \; \mintinline{tex}{\subsection*{Acknowledgement}}
\end{itemize}
\end{frame}

\begin{frame}[fragile]{Activity}
\begin{center}
	{\Large \textbf{Activity 4}}
\end{center}

\end{frame}



%--------------------------------------------------------

\section{Text Formatting}

\begin{frame}{Font Sizes and Colors}
%To change the font size in LaTeX \\
\begin{table}[]
	\begin{tabular}{ll}  
		\toprule
		\textbf{Commands}    & \textbf{Output} \\
		\midrule
		\mintinline{tex}{\tiny} & \tiny{LaTex}\\
		\mintinline{tex}{\scriptsize} & \scriptsize{LaTex}\\
		\mintinline{tex}{\footnotesize} & \footnotesize{LaTex}\\
		\mintinline{tex}{\small} & \small{LaTex}\\
		\mintinline{tex}{\normalsize} & {\normalsize LaTex}\\
		\mintinline{tex}{\large} & \large{LaTex}\\
		\mintinline{tex}{\Large} & \Large{LaTex}\\
		\mintinline{tex}{\LARGE} & \LARGE{LaTex}\\
		\mintinline{tex}{\huge} & {\huge LaTex}\\
		\mintinline{tex}{\Huge} & \Huge {LaTex}\\
		\bottomrule
	\end{tabular}%
	\label{tab:size}%
\end{table}
\end{frame}

\begin{frame}{Font Types and Style }
To change the font itself to different styles
\begin{table}[]
	\begin{tabular}{lll}  
		\toprule
		\textbf{Style} &\textbf{Commands}  & \textbf{Output} \\
		\midrule
		Bold & \mintinline{tex}{\textbf{LaTex}} & \textbf{LaTex}\\
		Italic &	\mintinline{tex}{\textit{LaTex}} & \textit{LaTex}\\
		Underline & \mintinline{tex}{\underline{LaTex}} & \underline{LaTex}\\
		Typewriter &	\mintinline{tex}{\texttt{LaTex}} & \texttt{LaTex}\\
		Sans-Serif & \mintinline{tex}{\textsf{LaTex}} & \textsf{LaTex}\\
		Serif (Roman) &	\mintinline{tex}{\textrm{LaTex}} & \textrm{LaTex}\\
		\bottomrule
	\end{tabular}%
	\label{tab:size}%
\end{table}

\end{frame}

\begin{frame}{Font Sizes and Colors}
To change text color use \mintinline{tex}{\usepackage{color}} or  \mintinline{tex}{\usepackage{xcolor}}
\begin{itemize}
	\item command:  \mintinline{tex}{\textcolor{color}{text}}
	\item Example: 
	\begin{itemize}
		\item \mintinline{tex}{\textcolor{red}{Hello} world} $\Rightarrow$ \textcolor{red}{Hello} world
		\item \mintinline{tex}{Hello \textcolor{blue}{world}} $\Rightarrow$ Hello \textcolor{blue}{world}
	\end{itemize}
	
\end{itemize}	

\end{frame}

\begin{frame}{Spacing}
LaTex treats any number of spaces as a single space.
\begin{itemize}
	\item Single new lines are treated as if there is no new line.
	\item Multiple blank lines are treated as a single new line or you may use \mintinline{tex}{\newline} or \mintinline{tex}{\\} command.
	\item You can force horizontal and vertical space using the \mintinline{tex}{\hspace{length}} and \mintinline{tex}{\vspace{length}}
	\begin{itemize}
		\item You have to give each command a length commands:
	\end{itemize} \mintinline{tex}{\hspace{0.1cm}}, \mintinline{tex}{\hspace{1in}} or 
	\mintinline{tex}{\hspace{10pt}}
	
	\item To  insert page breaks, use \mintinline{tex}{\clearpage} or \mintinline{tex}{\newpage}
	
\end{itemize}
\end{frame}

\begin{frame}[fragile]{Lists}
There are three list environments
\begin{itemize}
	\item itemize $\Rightarrow$ for a bullet list.
	\item enumerate $\Rightarrow$ for an ordered list and
	\item description $\Rightarrow$ for a descriptive list.
\end{itemize}
All lists follow the following format:
\begin{minted}{tex}
\begin{list_type}  
\item The first item 
\item The second item 
\item The third etc 
\end{list_type}
\end{minted}
\end{frame}

\begin{frame}[fragile]{Lists}
\begin{columns}[c] % the "c" option specifies center vertical alignment
\column{.5\textwidth} % column designated by a command
\begin{minted}{tex}
\begin{itemize}
\item The first item
\item The second item
\item  The third item
\end{itemize}
\end{minted}
\column{.5\textwidth}
\begin{itemize}
	\item The first item
	\item The second item
	\item  The third item
\end{itemize}
\end{columns}
\end{frame}

\begin{frame}[fragile]{Lists}
\begin{columns}[c] % the "c" option specifies center vertical alignment
\column{.5\textwidth} % column designated by a command
\begin{minted}{tex}
\begin{enumerate}
\item The first item
\item The second item
\item The third item
\end{enumerate}
\end{minted}
\column{.5\textwidth}
\begin{enumerate}
\item The first item
\item The second item
\item The third item
\end{enumerate}
\end{columns}
\end{frame}

\begin{frame}[fragile]{Lists}
The description list used to explain notations or terms
\footnotesize
\begin{minted}{tex}
\begin{description}
\item[Itemize] used for a bullet list.
\item[Enumerate] used for a ordered list.
\item[Description] used for a descriptive list.
\end{description}
\end{minted}

\alert{output}
\begin{description}
\item[Itemize] used for a bullet list.
\item[Enumerate] used for a ordered list.
\item[Description] used for a descriptive list.
\end{description}
\end{frame}


\begin{frame}[fragile]{Nested Lists}
\begin{columns}[c] % the "c" option specifies center vertical alignment
\column{.5\textwidth} % column designated by a command
\begin{minted}{tex}
\begin{itemize}
\item Item one
\begin{enumerate}
\item Subitem one
\item Subitem two
\end{enumerate}
\item Item two
\end{itemize}
\end{minted}
\column{.5\textwidth}
\begin{itemize}
\item Item one
\begin{enumerate}
\item Subitem one
\item Subitem two
\end{enumerate}
\item Item two
\end{itemize}
\end{columns}
\end{frame}


\begin{frame}[fragile]{Activity}
\begin{center}
{\Large \textbf{Activity 5}}
\end{center}

\end{frame}


\section{Cross-reference}\label{cross-ref}

\begin{frame}[fragile]{Cross-reference}
With the commands \mintinline{tex}{\label{key}  and \ref{key}} it is possible to refer to section numbers.
\begin{itemize}
	\item The command \mintinline{tex}{\label{key}} is used to set an identifier that is later used in the command \mintinline{tex}{\ref{key}} to set the reference.
\end{itemize}
\pause
\alert{Example}:\\
Create label: \framebox[1.0\width]{\mintinline{tex}{\section{Cross-Reference}\label{cross-ref}}}\\

Reference:
\framebox[1.0\width]{\mintinline{tex}{It is not difficult to refer to Section~\ref{cross-ref}}}\\

Output: \framebox[1.2\width]{It is not difficult to refer to Section \ref{cross-ref}}
\end{frame}

\begin{frame}[fragile]{Activity}
\begin{center}
{\Large \textbf{Activity 6}}
\end{center}

\end{frame}

%----------------------------------------------------

\section{Typesetting Mathematics}

\begin{frame}[fragile]{Math  mode}
The \texttt{amsmath} package is the backbone of using LaTex for typesetting math.
\begin{itemize}
	\item Include in preamble: \mintinline{tex}{\usepackage{amsmath}}
\end{itemize}
The math environment" comes in two different forms:
\begin{description}
	\item[Inline mode] $\Rightarrow$ format the math within existing lines of text.
	\item[Display mode] $\Rightarrow$ sets the math apart and centers it on the page.
\end{description}
\end{frame}

\begin{frame}[fragile]{Math  mode}
\framesubtitle{Inline mode}
Several options exist:
\begin{itemize}
	\item Use: \mintinline{tex}{\begin{math} x + y = 2 \end{math}} $\Rightarrow$ \begin{math} x + y = 2 \end{math}
	\item Use: \mintinline{tex}{\(x+y = 2\)}  $\Rightarrow$ \begin{math} x + y = 2 \end{math}
	\item Use: single dollar signs \mintinline{tex}{$x + y =2$} $\Rightarrow$ $x + y = 2$
\end{itemize}
\end{frame}



\begin{frame}[fragile]{Math  mode}
\framesubtitle{Inline mode}
Subscripts and superscripts in math mode are formed using the \mintinline{tex}{_} and the \mintinline{tex}{^}.\\
\alert{Example:}
\begin{center}
	\framebox[1.1\width]{$a_n = n^2 + 1$} $\Rightarrow$  \framebox[1.1\width]{\mintinline{tex}{$a_n = n^2 + 1$}}
\end{center}

When the subscript or superscript is more than one character,
you must wrap it in \{...\} to group it together.\\
\alert{Example:} 
\begin{center}
	\framebox[1.1\width]{\(y_{n + 1} = e^{n^2-1} + 1\) } $\Rightarrow$  \framebox[1.1\width]{\mintinline{tex}{$y_{n + 1} = e^{n^2-1} + 1$}}
\end{center}
\end{frame}

\begin{frame}[fragile]{Math  mode}
\framesubtitle{Inline mode}
Some common math symbols:
\centering
\scalebox{0.8}{
	\begin{tabular}{lp{5cm}}  
		\toprule
		\textbf{Symbol}    & \textbf{Output} \\
		\midrule
		\mintinline{tex}{\alpha,\beta,\lambda,\gamma,\theta,\mu etc} & $\alpha, \beta, \lambda, \gamma, \theta, \mu, $ etc \\
		\mintinline{tex}{\infty,\exists,\forall,\pm,\leq,\geq etc}. & $\infty,\exists,\forall,\pm,\leq,\geq$ etc\\
		\mintinline{tex}{\int_0^{\infinity},\sum_{i=1}^n,\prod_{n=1}^N } &  $\int_0^{\infty}, \sum_{i=1}^n, \prod_{n=1}^N$ etc\\
		\mintinline{tex}{\ldots, \cdots,\vdots,\colon etc} & $\ldots, \cdots,\vdots,\colon$etc \\
		\mintinline{tex}{\frac{x}{y}, \sqrt{x},\bar{x},\lim_{x \to \infty}} & $\frac{x}{y}, \sqrt{x},\bar{x}, \lim_{x \to \infty}$ etc\\
		\bottomrule
	\end{tabular}%
}
\label{tab:mathsymbol}%

\alert{More math symbols and formulas}: \href{https://en.wikibooks.org/wiki/LaTeX/Mathematics}{Latex Symbols} 
\end{frame}

\begin{frame}[fragile]{Math  mode}\framesubtitle{Common Math Formula}
\framebox[1.2\width]{$\frac{\partial y}{\partial x}$} $\Rightarrow$  \framebox[1.1\width]{\mintinline{tex}{$\frac{\partial y}{\partial x}$}}\\

\framebox[1.2\width]{$ \int_a^b f(x)\,dx $ } $\Rightarrow$  \framebox[1.1\width]{\mintinline{tex}{$ \int_a^b f(x)\,dx $ }}
\end{frame}

\begin{frame}[fragile]{Math  mode}
\framesubtitle{Display mode}
Several options exist:
\begin{itemize}
\item \mintinline{tex}{\begin{displaymath} x + y = 2 \end{displaymath}} $\Rightarrow$ \begin{displaymath} x + y = 2 \end{displaymath}
\item \mintinline{tex}{\[x+y = 2\]}  $\Rightarrow$ \[ x + y = 2 \]
\item \mintinline{tex}{$$x + y =2$$} $\Rightarrow$ $$x + y = 2$$
\end{itemize}
\end{frame}

\begin{frame}[fragile]{Math  mode}
\framesubtitle{Numbered Equation}
The \texttt{equation} environment: \mintinline{tex}{\begin{equation}...\end{equation}} creates a displayed formula and automatically generates
an equation number.\\
\alert{Example:}
\begin{equation}\int_{0}^{\pi}\sin x \,dx = 2\end{equation}
\begin{center}$\Downarrow$\end{center} 
\mint{tex}{\begin{equation}\int_{0}^{\pi}\sin x \,dx = 2\end{equation}} 
\end{frame}

\begin{frame}[fragile]{Math mode}
\framesubtitle{Referencing equations}
The amsmath package provides \mintinline{tex}{\eqref{key}} for referencing equations.\\
\alert{Example:}
\begin{columns}
\begin{column}{0.5\textwidth}
\begin{equation} \label{eq:1} 
\sum_{i=0}^{\infty} a_i x^i 
\end{equation}
equation~\eqref{eq:1} is a typical power series.
\end{column}
\hspace{5pt}\vrule\hspace{5pt}%
\begin{column}{0.5\textwidth}		
\begin{minted}{tex}
\begin{equation} \label{eq:1} 
\sum_{i=0}^{\infty} a_i x^i 
\end{equation}
equation~\eqref{eq:1} is a power series.
\end{minted}
\end{column}
\end{columns}	
\end{frame}

\begin{frame}[fragile]{Activity}
\begin{center}
	{\Large \textbf{Activity 7}}
\end{center}

\end{frame}

\begin{frame}[fragile]{Math  mode}
\framesubtitle{Multiple Equations}
The \mintinline{tex}{\begin{align}..\end{align}} environment is used group together several formulas or,  equations with more than one lines.
\alert{Example:}


\begin{columns}
	\begin{column}{0.5\textwidth}
		\begin{align}
		\alpha + \beta^2 &= 0 \\
		\log_{10}2\alpha &=e^{\beta}-1
		\end{align}
	\end{column}
	\hspace{5pt}\vrule\hspace{5pt}%
	\begin{column}{0.5\textwidth}		
\begin{minted}{tex}
\begin{align}
\alpha + \beta^2 &= 0 \\
\log_{10}2\alpha &=e^{\beta}-1
\end{align}
\end{minted}
	\end{column}
\end{columns}	


\end{frame}

\begin{frame}[fragile]{Math  mode}
\framesubtitle{Multiple Equations}
To align several formulas or equations with more than one lines.
\alert{Example:}
\begin{columns}
\begin{column}{0.5\textwidth}
	\begin{align*}
	y &=x^2 + 2x -1\\
	&=(x+1)(2x+1) \\
	&=(x+1)^2
	\end{align*}
\end{column}
\hspace{5pt}\vrule\hspace{5pt}%
\begin{column}{0.5\textwidth}		
	\begin{minted}{tex}
	\begin{align*}
	y &=x^2 + 2x -1\\
	&=(x+1)(2x+1) \\
	&=(x+1)^2
	\end{align*}
	\end{minted}
\end{column}
\end{columns}	

\end{frame}

\begin{frame}[fragile]{Math  mode}
\framesubtitle{Matrices and Array}
A basic matrix may be created using the \texttt{matrix} environment.\\

\alert{Plain Matrix}
\begin{columns}
	\begin{column}{0.5\textwidth}
		\[
		\begin{matrix}
		\alpha& \beta^{*}\\
		\gamma^{*}& \delta
		\end{matrix}
		\]
	\end{column}
	\hspace{5pt}\vrule\hspace{5pt}%
	\begin{column}{0.5\textwidth}		
		\begin{minted}{tex}
\[
\begin{matrix}
\alpha& \beta^{*}\\
\gamma^{*}& \delta
\end{matrix}
\]
		\end{minted}
	\end{column}
\end{columns}	

\end{frame}


\begin{frame}[fragile]{Math  mode}
\framesubtitle{Matrices and Array}

\alert{Bracketed matrix}; typically represents the matrix itself\\

\begin{columns}
\begin{column}{0.5\textwidth}
	\[
	\begin{bmatrix}
	\alpha& \beta^{*}\\
	\gamma^{*}& \delta
	\end{bmatrix}
	\]
\end{column}
\hspace{5pt}\vrule\hspace{5pt}%
\begin{column}{0.5\textwidth}		
\begin{minted}{tex}
\[
\begin{bmatrix}
\alpha& \beta^{*}
\gamma^{*}& \delta
\end{bmatrix}
\]
\end{minted}
\end{column}
\end{columns}	

\end{frame}


\begin{frame}[fragile]{Math  mode}
\framesubtitle{Matrices}

\alert{Parenthesized matrix}\\

\begin{columns}
\begin{column}{0.5\textwidth}	
\[
\begin{pmatrix}
\alpha& \beta^{*}\\
\gamma^{*}& \delta
\end{pmatrix}
\]
\end{column}
\hspace{5pt}\vrule\hspace{5pt}%
\begin{column}{0.5\textwidth}

\begin{minted}{tex}
\[
\begin{pmatrix}
\alpha& \beta^{*}\\
\gamma^{*}& \delta
\end{pmatrix}
\]
\end{minted}			

\end{column}
\end{columns}
\end{frame}

\begin{frame}[fragile]{Math mode}
\framesubtitle{Matrix}
\alert{Example: let type the following matrix}
\[ A_{m,n} =
\begin{pmatrix}
a_{1,1} & a_{1,2} & \cdots & a_{1,n} \\
a_{2,1} & a_{2,2} & \cdots & a_{2,n} \\
\vdots  & \vdots  & \ddots & \vdots  \\
a_{m,1} & a_{m,2} & \cdots & a_{m,n} 
\end{pmatrix}
\] 

\end{frame}


\begin{frame}[fragile]{Math mode}
\framesubtitle{Matrix}
\alert{Example: let type the following matrix}
\begin{center}
\begin{minted}{tex}
\[
A_{m,n} =
\begin{pmatrix}
a_{1,1} & a_{1,2} & \cdots & a_{1,n} \\
a_{2,1} & a_{2,2} & \cdots & a_{2,n} \\
\vdots  & \vdots  & \ddots & \vdots  \\
a_{m,1} & a_{m,2} & \cdots & a_{m,n} 
\end{pmatrix}
\] 
\end{minted}
\end{center}


\end{frame}

\begin{frame}[fragile]{Math mode}
\framesubtitle{The Case Environment}
The  \texttt{cases} environment allows the writing of piecewise functions.\\
Consider the following:
\begin{columns}
\begin{column}{0.5\textwidth}	
\[
f(x) = 
\begin{cases} 
x & \text{if } x \neq 0 \\
\frac{\sin x}{x}       & \text{otherwise} 
\end{cases}
\]
\end{column}
\hspace{5pt}\vrule\hspace{5pt}%
\begin{column}{0.5\textwidth}
\begin{minted}{tex}
\[
f(x) = 
\begin{cases} 
x & \text{if } x \neq 0 \\
\frac{\sin x}{x}& \text{otherwise} 
\end{cases}
\]
\end{minted}
\end{column}
\end{columns}	
\end{frame}

\begin{frame}[fragile]{Activity}
\begin{center}
{\Large \textbf{Activity 8}}
\end{center}

\end{frame}

%----------------------------------------------------

\section*{Collaborative Writing}
\begin{frame}[fragile]{Collaborative writing}
Most of your writing will be collaborative:
\begin{itemize}
	\item Often participants are distributed
	\item there are lots of ways to deal with this $\Rightarrow$ even when they are local, these techniques help.
\end{itemize}
Collaborative writing of documents requires a strong \alert{synchronisation} among authors.
\end{frame}

\begin{frame}{Collaborative writing}
Available modes
\begin{enumerate}
\item One person acts as editor, and incorporates changes: others  $\Rightarrow$ communicate proposed changes.
\begin{itemize}
	\item [-] lots of work for editor, but only they end up happy.
\end{itemize}
\item Token: one person has the \alert{token} (for all or part)
\begin{itemize}
	\item edit as please when have token pass it when $\Rightarrow$ finished (e.g. by email)
	\item [-] requires trust.
\end{itemize}
\item Truly distributed:
\begin{itemize}
	\item [+] all have access, and can edit.
	\item [+] conflicts are merged.
	\item [+] very powerful. 
	\item [-] but requires tools.
\end{itemize}
\end{enumerate}
\end{frame}

\begin{frame}{Collaborative writing models
	Truly distributed collaboration}
Require tools that support
\begin{itemize}
	\item Distributed access { e.g. Dropbox} and Revision control
\end{itemize}
Available options:\\[.50cm]
\emph{\href{https://www.dropbox.com/}{Dropbox} and/ or \href{https://www.box.com/home}{Box}}
	\begin{itemize}
		\item [-] often have a free limited plan.
		\item [+] save latex and the rest (e.g. accompanying code and data).
		\item [-] not a true versioning control system $\Rightarrow$ does not allow you to roll the article back to previous versions.
	\end{itemize}

\end{frame}


\begin{frame}{Collaborative writing models}
\emph{ Revision control System}
\begin{enumerate}[<+>]
	\item Online Payfor: e.g \href{https://www.overleaf.com/}{Overleaf} and \href{https://www.authorea.com/}{Authorea}
	\begin{itemize}
		\item [-] often have a free limited plan.
		\item [-] focussed on latex, not the rest (e.g. accompanying code and data).
	\end{itemize}
	\item free: standard open source tools eg git.
	\begin{itemize}
		\item [-] several option available: \href{https://www.mercurial-scm.org/}{Mercurial} and \href{https://git-scm.com/}{git}
		\item [+] support both latex and accompanying code and data.
		\item [+] offer control and advanced features like branch and merge.
		\item [-] steeper learning curve.
	\end{itemize}
\end{enumerate}
\end{frame}

\begin{frame}{Revision (or version) control}
Features:\begin{itemize}
\item Allow you to see all revisions of paper $\Rightarrow$ e.g. revert back to an old version if you don't like changes.
\item Trace activity $\Rightarrow$ volume also what changed, with comments.
\item Good for code,  LaTeX, and (some) data.
\end{itemize}
Cloud git services: \href{https://github.com/}{Github}, \href{https://bitbucket.org/}{BitBucket},\href{https://about.gitlab.com/}{Gitlab}.
\end{frame}

%-------------------------------------------------------------
\begin{frame}
\centering
\emph{THANK YOU}
\end{frame}



\end{document}

